%!TEX TS-program = xelatex
%! TEX encoding = UTF-8 Unicode

%========================================全文布局
\documentclass[twoside,openany]{book}
\usepackage[screen,paperheight=14.4cm,paperwidth=10.8cm,
left=2mm,right=2mm,top=2mm,bottom=5mm]{geometry}

\usepackage[]{microtype}
\usepackage{graphicx}
\usepackage{amssymb,amsmath}
\usepackage{booktabs}
\usepackage{titletoc}
\usepackage{titlesec}
\usepackage{tikz}
\usepackage{enumerate}
\usepackage{wallpaper}
\usepackage{indentfirst}
%========================================设置字体
\usepackage{ctex}
%\usepackage[CJKnumber]{xeCJK}
\usepackage{color}
\usepackage{xpinyin}
\xpinyinsetup{multiple={\color{red}}}
\setpinyin{子}{zi3}
\setCJKmainfont[BoldFont={Adobe Heiti Std R}]{Hiragino Sans GB W3}
\setCJKfamilyfont{kai}{Adobe Kaiti Std R}
\setCJKfamilyfont{hei}{Adobe Heiti Std R}
\setCJKfamilyfont{fsong}{Adobe Fangsong Std R}
\setmainfont{CMU Serif}

\newcommand{\kai}[1]{{\CJKfamily{kai}#1}}
\newcommand{\hei}[1]{{\CJKfamily{hei}#1}}
\newcommand{\fsong}[1]{{\CJKfamily{fsong}#1}}

\renewcommand\contentsname{目~录~}
\renewcommand\listfigurename{图~列~表~}
\renewcommand\listtablename{表~目~录~}
\usepackage{romannum}
%========================================章节样式
\titlecontents{chapter}
[0em]
{}
{\large\CJKfamily{hei}{}}
{}{\dotfill\contentspage}%用点填充
%
\titlecontents{section}
[4em]
{}
{\thecontentslabel\quad}
{}{\titlerule*{.}\contentspage}

\titleformat{\chapter}[display]
{\CJKfamily{fsong}\large\centering}
{\titlerule[1pt]%
	\filleft%
}
{-7ex}
{\Huge
	\filright}
[{\titlerule[1pt]}]

%========================================设置目录
\usepackage[setpagesize=false,
linkcolor=black,
colorlinks, %注释掉此项则交叉引用为彩色边框(将colorlinks和pdfborder同时注释掉)
pdfborder=001   %注释掉此项则交叉引用为彩色边框
]{hyperref}

\setlength{\parindent}{2em} %首行缩进
\linespread{1.4}              %行距
\setlength{\parskip}{12pt}    %段距

%========================================页眉页脚
\usepackage{fancyhdr}
\pagestyle{fancy}
\fancyhf{}
\fancyfoot{}
\fancyfoot[LE,RO]{\scriptsize\thepage}
\setlength{\footskip}{6pt}
%========================================标题作者
\title{中庸}
\author{\xpinyin*{孔伋}}
\date{}
\newcommand{\mt}[1]{\textbullet \textbf{#1}}
%========================================正文
\begin{document}
\TileSquareWallPaper{1}{TGTamber}%背景图片
\pagenumbering{arabic}
\maketitle
%\tableofcontents
	%\newpage
	
\zihao{4}
\noindent
	%\chapter*{封面语}\addcontentsline{toc}{chapter}{\large\CJKfamily{hei}封面语}
	
	
%\setcounter{chapter}{3}
%\chapter{学而第一}\label{ch1}
%卷一 学而第一
\begin{pinyinscope}
天命之谓性,\xpinyin{率}{shuai4}性之谓道,修道之谓教。

道也者,不可须臾离也;可离,非道也。
是故君子戒慎乎其所不睹,恐惧乎其所不闻。
莫\xpinyin{见}{xian4}乎隐,莫显乎微,故君子慎其独也。

喜怒哀乐之未发,谓之中;发而皆中节,谓之和;中也者,天下之大
本也;和也者,天下之达道也。致中和,天\xpinyin{地}{di4}位焉,万物育焉。

仲尼曰:“君子中庸,小人反中庸。君子之中庸也,君子而时中。小人之反中庸也,小人而无忌惮也。”

子曰:“中庸其至矣乎!民鲜能久矣!”

子曰:“道之不行也,我知之矣:知者过之,愚者不及也。道之不明也,我知之矣:贤者过之,不肖者不及也。人莫不饮食也,鲜能知味也。”

子曰:“道其不行矣夫!”

子曰:“舜其大\xpinyin{知}{zhi4}也与!舜\xpinyin{好}{hao4}问而\xpinyin{好}{hao4}察迩言,隐恶而扬善,执其两端,用其中于民,其斯以\xpinyin{为}{wei2}舜乎!”

子曰:“人皆曰‘\xpinyin{予}{yu2}\xpinyin{知}{zhi4}’,驱而纳诸罟擭护陷阱之中,而莫之知\xpinyin{辟}{bi4}也。人皆曰‘\xpinyin{予}{yu2}\xpinyin{知}{zhi4}’,择乎中庸,而不能\xpinyin{期}{ji1}月守也。”

子曰:“回之为人也,择乎中庸,得一善,则拳拳服膺,而弗失之矣。”

子曰:“天下国家可均也,爵禄可辞也,白刃可蹈也,中庸不可能也。”

子路问强。子曰:“南方之强与?北方之强与?抑而强与?
宽柔以教,不报无道,南方之强也,君子居之;衽金革,死而不厌,北方之强也,而强者居之。
故君子和而不流,强哉矫!
中立而不倚,强哉矫!
国有道,不变塞焉,强哉矫!
国无道,至死不变,强哉矫!”

子曰:“素隐行怪,后世有述焉,吾弗为之矣。
君子遵道而行,半涂而废,吾弗能已矣。
君子依乎中庸,遯世不见知而不悔,唯圣者能之。”

君子之道费而隐。夫妇之愚可以与知焉,及其至也,虽圣人亦有所不知焉。
夫妇之不肖,可以能行焉,及其至也,虽圣人亦有所不能焉。
天地之大也,人犹有所憾。
故君子语大,天下莫能载焉;语小,天下莫能破焉。

《诗》云:“鸢飞戾天,鱼跃于渊。”言其上下察也。

君子之道,造端乎夫妇,及其至也,察乎天地。

子曰:“道不远人,人之\xpinyin{为}{wei2}道而远人,不可以\xpinyin{为}{wei2}道。

《诗》云:‘伐柯伐柯,其则不远。’执柯以伐柯,睨而视之,犹以为远。
故君子以人治人,改而止。

忠恕违道不远,施诸己而不愿,亦勿施于人。

君子之道四,丘未能一焉,所求乎子以事父,未能也;所求乎臣以事君,未能也;所求乎弟以事兄,未能也;所求乎朋友先施之,未能也。
庸德之行,庸言之谨;有所不足,不敢不勉,有余不敢尽。
言顾行,行顾言,君子胡不慥慥尔!”

君子素其位而行,不愿乎其外。
素富贵,行乎富贵;素贫贱,行乎贫贱;素夷狄,行乎夷狄;素患难,行乎患难;君子无入而不自得焉。

在上位,不陵下;在下位,不援上。
正己而不求于人,则无怨。
上不怨天,下不尤人。故君子居易以\xpinyin{俟}{si4}命,小人行险以徼幸。

子曰:“射有似乎君子,失诸正鹄,反求诸其身。”

君子之道,\xpinyin{辟}{bi4}如行远,必自迩,\xpinyin{辟}{bi4}如登高,必自卑。

《诗》曰:“妻子好合,如鼓瑟琴。兄弟既翕,和乐且耽。宜尔室家,乐尔妻孥。”
子曰:“父母其顺矣乎!”

子曰:“鬼神之为德,其盛矣乎?!视之而弗见,听之而弗闻,体物而不可遗。
使天下之人,\xpinyin{齐}{zhai1}明盛服,以承祭祀,洋洋乎如在其上,如在其左右。

《诗》曰:‘神之格思,不可度思!矧可\xpinyin{射}{yi4}思!’夫微之显,诚之不可掩如此\xpinyin{夫}{fu2}!”

子曰:“舜其大孝也与!德为圣人,尊为天子,富有四海之内。
宗庙飨之,子孙保之。故大德必得其位,必得其禄,必得其名,必得其寿。
故天之生物,必因其材而笃焉,故栽者培之,倾者覆之。
《诗》曰:‘嘉乐君子,宪宪令德。宜民宜人,受禄于天,保佑命之,自天申之。’
故大德者必受命。”

子曰:“无忧者,其惟文王乎!以王季为父,以武王为子,父作之,子述之。
武王缵\xpinyin{大}{tai4}王、王季、文王之绪,\xpinyin{壹}{yi4}戎\xpinyin{衣}{yin1}而有天下,身不失天下之显名,尊为天子,富有四海之内,宗庙飨之,子孙保之。

武王末受命,周公成文武之德,追\xpinyin{王}{wang4}\xpinyin{大}{tai4}王、王季,上祀先公以天子之礼。
斯礼也,达乎诸侯、大夫,及士、庶人。
父为大夫,子为士,葬以大夫,祭以士;父为士,子为大夫,葬以士,祭以大夫。
\xpinyin{期}{ji4}之丧,达乎大夫;三年之丧,达乎天子;父母之丧,无贵贱,一也。”

子曰:“武王、周公,其达孝矣乎!
夫孝者,善继人之志,善述人之事者也。
春秋,修其祖庙,陈其宗器,设其裳衣,荐其时食。
宗庙之礼,所以序昭穆也;序爵,所以辨贵贱也;序事,所以辨贤也;旅酬下为上,所以逮贱也;燕毛,所以序齿也。
践其位,行其礼,奏其乐,敬其所尊,爱其所亲,事死如事生,事亡如事存,孝之至也。
郊社之礼,所以事上帝也;宗庙之礼,所以祀乎其先也。
明乎郊社之礼、禘尝之义,治国其如示诸掌乎!”

哀公问政。子曰:“文武之政,布在方策。其人存,则其政举;其人亡,则其政息。
人道敏政,地道敏树。夫政也者,蒲卢也。

“故为政在人,取人以身,修身以道,修道以仁。
仁者,人也,亲\xpinyin{亲}{qin}为大。义者,宜也,尊贤为大。
亲\xpinyin{亲}{qin}之\xpinyin{杀}{shai4},尊贤之等,礼所生也。
在下位,不获乎上,民不可得而治矣!
故君子不可以不修身;思修身,不可以不事亲;思事亲,不可以不知人;思知人,不可以不知天。

“天下之达道五,所以行之者三。
曰:‘君臣也,父子也,夫妇也,昆弟也,朋友之交也。’
五者,天下之达道也。
‘\xpinyin{知}{zhi4},仁,勇’三者,天下之达德也,所以行之者一也。

“或生而知之,或学而知之,或困而知之,及其知之一也。
或安而行之,或利而行之,或勉强而行之,及其成功一也。”

子曰:“\xpinyin{好}{hao4}学近乎\xpinyin{知}{zhi4},力行近乎仁,知耻近乎勇。
知斯三者,则知所以修身;知所以修身,则知所以治人;知所以治人,则知所以治天下国家矣。

“凡\xpinyin{为}{wei2}天下国家有九经,曰:修身也,尊贤也,亲\xpinyin{亲}{qin}也,敬大臣也,体群臣也,子庶民也,来百工也,柔远人也,怀诸侯也。

“修身,则道立;尊贤,则不惑;亲\xpinyin{亲}{qin},则诸父昆弟不怨;敬大臣,则不眩;体群臣,则士之报礼重;子庶民,则百姓劝;来百工,则财用足;柔远人,则四方归之;怀诸侯,则天下畏之。

“\xpinyin{齐}{zhai1}明盛服,非礼不动,所以修身也;去谗远色,贱货而贵德,所以劝贤也;尊其位,重其禄,同其\xpinyin{好}{hao4}\xpinyin{恶}{wu4},所以劝亲\xpinyin{亲}{qin}也;官盛任使,所以劝大臣也;忠信重禄,所以劝士也;时使薄敛,所以劝百姓也;日\xpinyin{省}{xing3}月试,\xpinyin{既}{xi4}廪\xpinyin{称}{chen4}事,所以劝百工也;送往迎来,嘉善而矜不能,所以柔远人也;继绝世,举废国,治乱持危,朝聘以时,厚往而薄来,所以怀诸侯也。
凡为天下国家有九经,所以行之者一也。

“凡事豫则立,不豫则废。言前定,则不跲;事前定,则不困;行前定,则不疚;道前定,则不穷。

“在下位不获乎上,民不可得而治矣。
获乎上有道,不信乎朋友,不获乎上矣;信乎朋友有道,不顺乎亲,不信乎朋友矣;顺乎亲有道,反诸身不诚,不顺乎亲矣;诚身有道,不明乎善,不诚乎身矣。

“诚者,天之道也;诚之者,人之道也。
诚者,不勉而中,不思而得,从容中道,圣人也。
诚之者,择善而固执之者也。

“博学之,审问之,慎思之,明辨之,笃行之。
有弗学,学之弗能,弗措也;有弗问,问之弗知,弗措也;有弗思,思之弗得,弗措也;有弗辨,辨之弗明,弗措也;有弗行,行之弗笃,弗措也。
人一能之,己百之,人十能之,己千之。
果能此道矣,虽愚必明,虽柔必强。”

自诚明,谓之性;自明诚,谓之教;诚则明矣,明则诚矣。

唯天下至诚\xpinyin{为}{wei2}能\xpinyin{尽}{jin4}其性。能\xpinyin{尽}{jin4}其性,则能\xpinyin{尽}{jin4}人之性;能\xpinyin{尽}{jin4}人之性,则能\xpinyin{尽}{jin4}物之性;能\xpinyin{尽}{jin4}物之性,则可以赞天\xpinyin{地}{di4}之化育;可以赞天\xpinyin{地}{di4}之化育,则可以与天地\xpinyin{参}{san1}矣。

其次致曲。曲能有诚,诚则形,形则\xpinyin{著}{zhu4},\xpinyin{著}{zhu4}则明,明则动,动则变,变则化。唯天下至诚\xpinyin{为}{wei2}能化。

至诚之道,可以前知。国家将\xpinyin{兴}{xing1},必有祯祥;国家将亡,必有妖孽。
\xpinyin{见}{xian4}乎蓍龟,动乎四体。
祸福将至,善,必先知之;不善,必先知之。故至诚如神。

诚者自成也,而道自道也。诚者物之终始,不诚无物。是故君子诚之\xpinyin{为}{wei2}贵。
诚者,非自成己而已也,所以成物也。成己,仁也;成物,\xpinyin{知}{zhi4}也。
性之德也,合外内之道也,故时措之宜也。

故至诚无息,不息则久,久则征,征则悠远,悠远则博厚,博厚则高明。
博厚,所以,\xpinyin{载}{zai3}物也;高明,所以覆物也;悠久,所以成物也。
博厚配\xpinyin{地}{di4},高明配天,悠久无疆。
如此者,不\xpinyin{见}{xian4}而章,不动而变,无\xpinyin{为}{wei2}而成。

天地之道,可一言而尽也。
其\xpinyin{为}{wei2}物不贰,则其生物不测。
天地之道:博也,厚也,高也,明也,悠也,久也。
今夫天,斯昭昭之多,及其无穷也,日月星辰系焉,万物覆焉。
今夫\xpinyin{地}{di4},一撮土之多,及其广厚,\xpinyin{载}{zai3}华岳而不重,振河海而不泄,万物\xpinyin{载}{zai3}焉。
今夫山,一卷石之多,及其广大,草木生之,禽兽居之,宝\xpinyin{藏}{zang4}\xpinyin{兴}{xing1}焉。
今夫水,一勺之多,及其不测,鼋、鼍、蛟、龙、鱼、鳖生焉,货财殖焉。

《诗》曰:“惟天之命,于穆不已!”
盖曰:“天之所以\xpinyin{为}{wei2}天也。于乎不显,文王之德之纯!”
盖曰:“文王之所以\xpinyin{为}{wei2}文也,纯亦不已。”

大哉圣人之道!洋洋乎,发育万物,峻极于天。
优优大哉!礼仪三百,威仪三千。
待其人然后行。故曰:“苟不至德,至道不凝焉。”
故君子尊德性而道问学,致广大而尽精微,极高明而道中庸。
温故而知新,敦厚以崇礼。
是故居上不骄,\xpinyin{为}{wei2}下不倍。
国有道,其言足以\xpinyin{兴}{xing1};国无道,其默足以容。
诗曰:“既明且哲,以保其身。”其此之谓与!

子曰:“愚而好自用,贱而好自专,生乎今之世,反古之道。
如此者,灾及其身者也。”

非天子,不议礼,不制度,不考文。
今天下车同轨,书同文,行同伦。
虽有其位,苟无其德,不敢作礼\xpinyin{乐}{yue4}焉;虽有其德,苟无其位,亦不敢作礼乐焉。

子曰:“吾说夏礼,杞不足徵也。吾学殷礼,有宋存焉。吾学周礼,今用之,吾从周。”

王天下有三重焉,其寡过矣乎!上焉者虽善,无徵,无徵不信,不信民弗从;
下焉者虽善,不尊,不尊不信,不信民弗从。
故君子之道:本诸身,徵诸庶民,考诸三王而不\xpinyin{缪}{miu4},建诸天\xpinyin{地}{di4}而不悖,质诸鬼神而无疑,百世以\xpinyin{俟}{si42}圣人而不惑。
质诸鬼神而无疑,知天也;百世以\xpinyin{俟}{si42}圣人而不惑,知人也。

是故,君子动而世\xpinyin{为}{wei2}天下道,行而世\xpinyin{为}{wei2}天下法,言而世\xpinyin{为}{wei2}天下则。
远之则有望,近之则不厌。

《诗》曰:“在彼无\xpinyin{恶}{wu4},在此无\xpinyin{射}{yi4}。庶几夙夜,以永终誉!”
君子未有不如此,而蚤有誉于天下者也。

仲尼祖述尧舜,宪章文武,上律天时,下袭水土。
辟如天\xpinyin{地}{di4}之无不持载,无不覆\xpinyin{帱}{dao4};辟如四时之错行,如日月之代明。
万物并育而不相害,道并行而不相悖,小德川流,大德敦化,此天\xpinyin{地}{di4}之所以\xpinyin{为}{wei2}大也。

唯天下至圣,\xpinyin{为}{wei2}能聪明睿知,足以有临也;宽裕温柔,足以有容也;发强刚毅,足以有执也;\xpinyin{齐}{zhai1}庄中正,足以有敬也;文理密察,足以有别也。

溥博渊泉,而时出之。溥博如天,渊泉如渊。\xpinyin{见}{xian4}而民莫不敬,言而民莫不信,行而民莫不\xpinyin{说}{yue4}。

是以声名洋溢乎中国,施及蛮貊。舟车所至,人力所通,天之所覆,地之所载,日月所照,霜露所\xpinyin{队}{zhui4},凡有血气者,莫不尊亲。
故曰:“配天”。

唯天下至诚,\xpinyin{为}{wei2}能经纶天下之大经,立天下之大本,知天\xpinyin{地}{di4}之化育。
夫焉有所倚?肫肫其仁!渊渊其渊!浩浩其天!苟不固聪明圣知达天德者,其孰能知之?

《诗》曰:“\xpinyin{衣}{yi4}锦尚\xpinyin{絅}{jiong3}。”
\xpinyin{恶}{wu4}其文之\xpinyin{著}{zhu4}也。
故君子之道,暗然而日章;小人之道,\xpinyin{的}{di4}然而日亡。
君子之道:淡而不厌,简而文,温而理,知远之近,知风之自,知微之显,可与入德矣。

《诗》云:“潜虽伏矣,亦孔之昭!”
故君子内\xpinyin{省}{xing3}不疚,无\xpinyin{恶}{wu4}于志。
君子之所不可及者,其唯人之所不见乎!

《诗》云:“\xpinyin{相}{xiang4}在尔室,尚不愧于屋漏。”
故君子不动而敬,不言而信。

《诗》曰:“奏假无言,时靡有争。”
是故君子不赏而民劝,不怒而民威于鈇钺。

《诗》曰:“\xpinyin{不}{pi3}显惟德!百\xpinyin{辟}{bi4}其刑之。”
是故君子笃恭,而天下平。

《诗》云:“予怀明德,不大声以色。”
子曰:“声色之于以化民,末也。”
《诗》曰:“德\xpinyin{輶}{you2}如毛。”
毛犹有伦。“上天之载,无声无\xpinyin{臭}{xiu4}。”至矣!
\end{pinyinscope}
\end{document}