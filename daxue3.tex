%!TEX TS-program = xelatex
%! TEX encoding = UTF-8 Unicode

%========================================全文布局
\documentclass[twoside,openany]{book}
\usepackage[screen,paperheight=14.4cm,paperwidth=10.8cm,
left=2mm,right=2mm,top=2mm,bottom=5mm]{geometry}

\usepackage[]{microtype}
\usepackage{graphicx}
\usepackage{amssymb,amsmath}
\usepackage{booktabs}
\usepackage{titletoc}
\usepackage{titlesec}
\usepackage{tikz}
\usepackage{enumerate}
\usepackage{wallpaper}
\usepackage{indentfirst}
%========================================设置字体
\usepackage{ctex}
\usepackage{textcircled}
\usepackage{NumGo}
\usepackage{color}
\usepackage{xpinyin}
%\xpinyinsetup{multiple={\color{red}}}
\setCJKmainfont[BoldFont={Adobe Heiti Std R}]{Hiragino Sans GB W3}
\setCJKfamilyfont{kai}{Adobe Kaiti Std R}
\setCJKfamilyfont{hei}{Adobe Heiti Std R}
\setCJKfamilyfont{fsong}{Adobe Fangsong Std R}
\setmainfont{CMU Serif}

\newcommand{\kai}[1]{{\CJKfamily{kai}#1}}
\newcommand{\hei}[1]{{\CJKfamily{hei}#1}}
\newcommand{\fsong}[1]{{\CJKfamily{fsong}#1}}

\renewcommand\contentsname{目~录~}
\renewcommand\listfigurename{图~列~表~}
\renewcommand\listtablename{表~目~录~}
\usepackage{romannum}
%========================================章节样式
\titlecontents{chapter}
[0em]
{}
{\large\CJKfamily{hei}{}}
{}{\dotfill\contentspage}%用点填充
%
\titlecontents{section}
[4em]
{}
{\thecontentslabel\quad}
{}{\titlerule*{.}\contentspage}

\titleformat{\chapter}[display]
{\CJKfamily{fsong}\large\centering}
{\titlerule[1pt]%
	\filleft%
}
{-7ex}
{\Huge
	\filright}
[{\titlerule[1pt]}]

%========================================设置目录
\usepackage[setpagesize=false,
linkcolor=black,
colorlinks, %注释掉此项则交叉引用为彩色边框(将colorlinks和pdfborder同时注释掉)
pdfborder=001   %注释掉此项则交叉引用为彩色边框
]{hyperref}

\setlength{\parindent}{2em} %首行缩进
\linespread{1.4}              %行距
\setlength{\parskip}{12pt}    %段距

%========================================页眉页脚
\usepackage{fancyhdr}
\pagestyle{fancy}
\fancyhf{}
\fancyfoot{}
\fancyfoot[LE,RO]{\scriptsize\thepage}
\setlength{\footskip}{6pt}
%========================================标题作者
\title{大学}
\author{\xpinyin{曾}{zeng1}\xpinyin{参}{shen1}}
\date{}
\newcommand{\mt}[1]{\textbullet \textbf{#1}}
%========================================正文
\begin{document}
\TileSquareWallPaper{1}{TGTamber}%背景图片
\pagenumbering{arabic}
\maketitle
%\tableofcontents
	%\newpage
	
\zihao{4}
\noindent
	%\chapter*{封面语}\addcontentsline{toc}{chapter}{\large\CJKfamily{hei}封面语}
	
	
%\setcounter{chapter}{3}
%\chapter{学而第一}\label{ch1}
%卷一 学而第一
\begin{pinyinscope}
%\chapter{第一章}\label{ch01}
%\textsuperscript{\textcircled{\small 1}}
%{\small\tikztextcircle[black][white]{1}}
\oblack{2}
大学之道:在明明德,在\xpinyin{亲}{xin1}民,在止于至善。
知止而後有定,定而後能静,静而後能安,安而後能虑,虑而後能得。
物有本末,事有终始。知所先後,则近道矣。

古之欲明明德于天下者,先治其国;欲治其国者,先齐其家;欲齐其家者,先修其身;欲修其身者,先正其心;欲正其心者,先诚其意;欲诚其意者,先致其知;致知在格物。
物格而後知至,知至而後意诚,意诚而後心正,心正而後身修,身修而後家齐,家齐而後国治,国治而後天下平。

自天\xpinyin{子}{zi3}以至于庶人,壹是皆以修身\xpinyin{为}{wei2}本。
其本乱而末治者否矣;其所厚者薄,而其所薄者厚,未之有也。

此谓知本,此谓知之至也。

%\chapter{第二章}\label{ch02}
%\textsuperscript{\tikztextcircle[black][white]{\small 2}}
\oblack{3}
所谓“诚其意”者,毋自欺也。
如\xpinyin{恶}{wu4}恶\xpinyin{臭}{xiu4},如\xpinyin{好}{hao4}好色,此之谓自\xpinyin{谦}{qie4}。
故君\xpinyin{子}{zi3}必慎其独也。
小人闲居\xpinyin{为}{wei2}不善,无所不至;见君\xpinyin{子}{zi3}而後\xpinyin{厌}{yan3}然,掩其不善而\xpinyin{著}{zhu4}其善;人之视己,如见其肺肝然,则何益矣?
此谓诚于中、形于外。
故君\xpinyin{子}{zi3}必慎其独也。

\xpinyin{曾}{zeng1}\xpinyin{子}{zi3}曰:“十目所视,十手所指,其严乎!”
富润屋,德润身,心广体\xpinyin{胖}{pan2},故君\xpinyin{子}{zi3}必诚其意。

%\chapter{第三章}\label{ch03}
%\textsuperscript{\textcircled{\small 3}}
\oblack{4}
诗云:“瞻彼淇\xpinyin{澳}{yu4},菉竹猗猗;有斐君\xpinyin{子}{zi3},如切如磋,如琢如磨;瑟兮僴兮,赫兮喧兮;有斐君\xpinyin{子}{zi3},终不可諠兮!”
如切如磋者,道学也;如琢如磨者,自修也;瑟兮僴兮者,恂慄也;赫兮喧兮者,威仪也;有斐君\xpinyin{子}{zi3},终不可諠兮者,道盛德至善,民之不能忘也。
诗云:“\xpinyin{於}{wu1}\xpinyin{戏}{hu1}!前王不忘。”
君子贤其贤而亲其亲,小人乐其乐而利其利,此以\xpinyin{没}{mo4}世不忘也。

%\chapter{第四章}\label{ch04}
%\textsuperscript{\textcircled{\scriptsize 4}}
\oblack{5}
康诰曰:“克明德。”
太甲曰:“顾諟天之明命。”
帝典曰:“克明峻德。”
皆自明也。

汤之盘铭曰:“苟日新,日日新,又日新。”
康诰曰:“作新民。”
诗曰:“周虽旧邦,其命维新。”
是故,君\xpinyin{子}{zi3}无所不用其极。

诗云:“邦畿千里,维民所止。”
诗云:“缗蛮黄鸟,止于丘隅。”
\xpinyin{子}{zi3}曰:“于止,知其所止,可以人而不如鸟乎?”

诗云:“穆穆文王,\xpinyin{於}{wu1}\xpinyin{缉}{qi1}熙敬止!”
\xpinyin{为}{wei2}人君,止于仁;\xpinyin{为}{wei2}人臣,止于敬;\xpinyin{为}{wei2}人\xpinyin{子}{zi3},止于孝;\xpinyin{为}{wei2}人父,止于慈;与国人交,止于信。

\xpinyin{子}{zi3}曰:“听讼,吾犹人也;必也使无讼乎!”
无情者不得尽其辞,大畏民志;此谓知本。

%\chapter{第五章}\label{ch05}
\oblack{6}
所谓“修身在正其心”者,身有所忿懥,则不得其正;有所恐惧,则不得其正;有所\xpinyin{好}{hao4}乐,则不得其正;有所忧患,则不得其正。
心不在焉,视而不见,听而不闻,食而不知其味。
此谓“修身在正其心”。

%\chapter{第六章}\label{ch06}
\oblack{7}
所谓“齐其家在修其身”者,人之其所亲爱而辟焉,之其所贱\xpinyin{恶}{wu4}而辟焉,之其所畏敬而辟焉,之其所哀矜而辟焉,之其所\xpinyin{敖}{ao4}惰而辟焉。
故\xpinyin{好}{hao4}而知其恶,\xpinyin{恶}{wu4}而知其美者,天下鲜矣。
故谚有之曰:“人莫知其\xpinyin{子}{zi3}之恶,莫知其苗之硕。”
此谓身不修,不可以齐其家。

%\chapter{第七章}\label{ch07}
\oblack{8}
所谓“治国必先齐其家”者,其家不可\xpinyin{教}{jiao1},而能\xpinyin{教}{jiao1}人者,无之。
故君\xpinyin{子}{zi3}不出家,而成教于国。
孝者,所以事君也;\xpinyin{弟}{ti4}者,所以事长也;慈者,所以使众也。
康诰曰:“如保赤\xpinyin{子}{zi3}。”
心诚求之,虽不\xpinyin{中}{zhong4},不远矣。
未有学养\xpinyin{子}{zi3},而後嫁者也。

一家仁,一国\xpinyin{兴}{xing1}仁;一家让,一国\xpinyin{兴}{xing1}让;一人贪戾,一国作乱;其机如此。
此谓一言偾事,一人定国。
尧、舜\xpinyin{帅}{shuai4}天下以仁,而民从之;桀、纣\xpinyin{帅}{shuai4}天下以暴,而民从之。
其所令反其所好,而民不从。
是故,君\xpinyin{子}{zi3}有诸己,而後求诸人;无诸己,而後非诸人。
所藏乎身不恕,而能喻诸人者,未之有也。
故治国在齐其家。

诗云:“桃之夭夭,其叶蓁蓁,之\xpinyin{子}{zi3}于归,宜其家人。”
宜其家人,而後可以\xpinyin{教}{jiao1}国人。
诗云:“宜兄宜弟。”
宜兄宜弟,而後可以\xpinyin{教}{jiao1}国人。
诗云:“其仪不忒,正是四国。”
其\xpinyin{为}{wei2}父\xpinyin{子}{zi3}兄弟足法,而後民法之也。
此谓治国在齐其家。

%此谓知本,此谓知之至也。
%
%所谓致知在格物者,言欲致吾之知,在即物而穷其理也。盖人心之灵莫不有知,而天下之物莫不有理,唯于理有未穷,故其知又不尽也,是以《大学》始教,必使学者即凡于天下之物,莫不因其已知之理而益穷之,以求至乎其极。至于用力之久,而一旦豁然贯通焉,则众物之表里精粗无不到,而吾心之全体大用无不明矣。此谓物格,此谓知之至也。

%\chapter{第八章}\label{ch08}
\oblack{9}
所谓“平天下在治其国”者,上老老而民\xpinyin{兴}{xing1}孝;上长长而民\xpinyin{兴}{xing1}\xpinyin{弟}{ti4};上恤孤而民不倍。
是以君\xpinyin{子}{zi3}有\xpinyin{絜}{xie2}矩之道也。
所\xpinyin{恶}{wu4}于上,毋以使下;所\xpinyin{恶}{wu4}于下,毋以事上;所\xpinyin{恶}{wu4}于前,毋以先後;所\xpinyin{恶}{wu4}于後,毋以从前;所\xpinyin{恶}{wu4}于右,毋以交于左;所\xpinyin{恶}{wu4}于左,毋以交于右;此之谓\xpinyin{絜}{xie2}矩之道。
诗云:“乐只君\xpinyin{子}{zi3},民之父母。”
民之所\xpinyin{好}{hao4}\xpinyin{好}{hao4}之,民之所\xpinyin{恶}{wu4}\xpinyin{恶}{wu4}之,此之谓民之父母。
诗云:“节彼南山,维石岩岩,赫赫师尹,民具尔瞻。”
有国者不可以不慎,辟,则为天下僇矣。

%\chapter{第九章}\label{ch09}
\oblack{10}
诗云:“殷之未丧师,克配上帝;仪监于殷,峻命不易。”
道得众,则得国;失众,则失国。
是故君\xpinyin{子}{zi3}先慎乎德,有德此有人,有人此有土,有土此有财,有财此有用。
德者,本也;财者,末也。
外本内末,争民施夺。
是故财聚则民散,财散则民聚。
是故言悖而出者,亦悖而入;货悖而入者,亦悖而出。
康诰曰:“惟命不于常。”
道善则得之,\xpinyin{不}{bu2}善则失之矣。
楚书曰:“楚国无以\xpinyin{为}{wei2}宝,惟善以\xpinyin{为}{wei2}宝。”
舅犯曰:“亡人无以\xpinyin{为}{wei2}宝,仁亲以\xpinyin{为}{wei2}宝。”

%\chapter{第十章}\label{ch10}
\oblack{11}
秦誓曰:“若有一个臣,断断兮,无他技;其心休休焉,其如有容焉。
人之有技,若己有之;人之彦圣,其心\xpinyin{好}{hao4}之;不啻若自其口出,寔能容之,以能保我\xpinyin{子}{zi3}孙黎民,尚亦有利哉!
人之有技,媢嫉以\xpinyin{恶}{wu4}之;人之彦圣,而违之俾不通;寔不能容,以不能保我\xpinyin{子}{zi3}孙黎民,亦曰殆哉!”
唯仁人放流之,\xpinyin{迸}{bing3}诸四夷,不与同中国。
此谓“唯仁人\xpinyin{为}{wei2}能爱人,能\xpinyin{恶}{wu4}人。”
见贤而不能举,举而不能先,命也;见\xpinyin{不}{bu2}善而不能退,退而不能远,过也。
\xpinyin{好}{hao4}人之所\xpinyin{恶}{wu4},\xpinyin{恶}{wu4}人之所\xpinyin{好}{hao4},是谓拂人之性,菑必\xpinyin{逮}{dai4}\xpinyin{夫}{fu2}身。
是故君\xpinyin{子}{zi3}有大道,必忠信以得之,骄泰以失之。

%\chapter{第十一章}\label{ch11}
\oblack{12}
生财有大道:生之者众,食之者寡;\xpinyin{为}{wei2}之者疾,用之者舒;则财恒足矣。
仁者以财发身,不仁者以身发财。
未有上\xpinyin{好}{hao4}仁,而下\xpinyin{不}{bu2}\xpinyin{好}{hao4}义者也;未有\xpinyin{好}{hao4}义,其事不终者也;未有府库财,非其财者也。

孟献\xpinyin{子}{zi3}曰:“\xpinyin{畜}{xu4}马\xpinyin{乘}{sheng4},不察于鸡豚;伐冰之家,\xpinyin{不}{bu2}\xpinyin{畜}{xu4}牛羊,百\xpinyin{乘}{sheng4}之家,\xpinyin{不}{bu2}\xpinyin{畜}{xu4}聚敛之臣;与其有聚敛之臣,\xpinyin{宁}{ning4}有盗臣。”
此谓国不以利\xpinyin{为}{wei2}利,以义\xpinyin{为}{wei2}利也。
长国家而务财用者,必自小人矣,彼\xpinyin{为}{wei2}善之。
小人之使\xpinyin{为}{wei2}国家,菑害并至,虽有善者,亦无如之何矣。
此谓“国不以利\xpinyin{为}{wei2}利,以义\xpinyin{为}{wei2}利”也。
\end{pinyinscope}
\end{document}