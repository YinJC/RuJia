%!TEX TS-program = xelatex
%! TEX encoding = UTF-8 Unicode

%========================================全文布局
\documentclass[twoside,openany]{book}
\usepackage[screen,paperheight=14.4cm,paperwidth=10.8cm,
left=2mm,right=2mm,top=2mm,bottom=5mm]{geometry}

\usepackage[]{microtype}
\usepackage{graphicx}
\usepackage{amssymb,amsmath}
\usepackage{booktabs}
\usepackage{titletoc}
\usepackage{titlesec}
\usepackage{tikz}
\usepackage{enumerate}
\usepackage{wallpaper}
\usepackage{indentfirst}
%========================================设置字体
\usepackage{ctex}
%\usepackage[CJKnumber]{xeCJK}
\usepackage{xpinyin}
\setCJKmainfont[BoldFont={Adobe Heiti Std R}]{Hiragino Sans GB W3}
\setCJKfamilyfont{kai}{Adobe Kaiti Std R}
\setCJKfamilyfont{hei}{Adobe Heiti Std R}
\setCJKfamilyfont{fsong}{Adobe Fangsong Std R}

\newcommand{\kai}[1]{{\CJKfamily{kai}#1}}
\newcommand{\hei}[1]{{\CJKfamily{hei}#1}}
\newcommand{\fsong}[1]{{\CJKfamily{fsong}#1}}

\renewcommand\contentsname{目~录~}
\renewcommand\listfigurename{图~列~表~}
\renewcommand\listtablename{表~目~录~}
\usepackage{romannum}
%========================================章节样式
\titlecontents{chapter}
[0em]
{}
{\large\CJKfamily{hei}{}}
{}{\dotfill\contentspage}%用点填充
%
\titlecontents{section}
[4em]
{}
{\thecontentslabel\quad}
{}{\titlerule*{.}\contentspage}

\titleformat{\chapter}[display]
{\CJKfamily{fsong}\large\centering}
{\titlerule[1pt]%
	\filleft%
}
{-7ex}
{\Huge
	\filright}
[{\titlerule[1pt]}]

%========================================设置目录
\usepackage[setpagesize=false,
linkcolor=black,
colorlinks, %注释掉此项则交叉引用为彩色边框(将colorlinks和pdfborder同时注释掉)
pdfborder=001   %注释掉此项则交叉引用为彩色边框
]{hyperref}

\setlength{\parindent}{2em} %首行缩进
\linespread{1.2}              %行距
\setlength{\parskip}{15pt}    %段距

%========================================页眉页脚
\usepackage{fancyhdr}
\pagestyle{fancy}
\fancyhf{}
\fancyfoot{}
\fancyfoot[LE,RO]{\scriptsize\thepage}
\setlength{\footskip}{6pt}
%========================================标题作者
\title{论语}
\author{ }
\date{}
\newcommand{\mt}[1]{\textbullet \textbf{#1}}
%========================================正文
\begin{document}
\TileSquareWallPaper{1}{TGTamber}%背景图片
\pagenumbering{arabic}
\maketitle
\tableofcontents
	%\newpage
	
\zihao{4}
\noindent
	%\chapter*{封面语}\addcontentsline{toc}{chapter}{\large\CJKfamily{hei}封面语}
	
	
%\setcounter{chapter}{3}
\chapter{学而第一}\label{ch1}
%卷一 学而第一

子曰:“学而时习之,不亦说乎?有朋自远方来,不亦乐乎?人不知而不愠,不亦君子乎?”

有子曰:“其为人也孝弟,而好犯上者,鲜矣;不好犯上,而好作乱者,未之有也。君子务本,本立而道生。孝弟也者,其为仁之本与!”

子曰:“巧言令色,鲜矣仁!”

曾子曰:“吾日三省吾身:为人谋而不忠乎?与朋友交而不信乎?传不习乎?”

子曰:“道千乘之国:敬事而信,节用而爱人,使民以时。”

子曰:“弟子入则孝,出则弟,谨而信,泛爱众,而亲仁。行有余力,则以学文。”

子夏曰:“贤贤易色,事父母能竭其力,事君能致其身,与朋友交言而有信。虽曰未学,吾必谓之学矣。”

子曰:“君子不重则不威,学则不固。主忠信。无友不如己者。过则勿惮改。”

曾子曰:“慎终追远,民德归厚矣。”

子禽问于子贡曰:“夫子至于是邦也,必闻其政,求之与?抑与之与?”子贡曰:“夫子温、良、恭、俭、让以得之。夫子之求之也,其诸异乎人之求之与?”

子曰:“父在,观其志;父没,观其行;三年无改于父之道,可谓孝矣。”

有子曰:“礼之用,和为贵。先王之道斯为美,小大由之。有所不行,知和而和,不以礼节之,亦不可行也。”

有子曰:“信近于义,言可复也;恭近于礼,远耻辱也;因不失其亲,亦可宗也。”

子曰:“君子食无求饱,居无求安,敏于事而慎于言,就有道而正焉,可谓好学也已。”

子贡曰:“贫而无谄,富而无骄,何如?”子曰:“可也。未若贫而乐,富而好礼者也。”子贡曰:“诗云:‘如切如磋,如琢如磨。’其斯之谓与?”子曰:“赐也,始可与言诗已矣!告诸往而知来者。”

子曰:“不患人之不己知,患不知人也。”


\chapter{为政第二}\label{ch2}
%卷二 为政第二

子曰:“为政以德,譬如北辰,居其所而众星共之。”

子曰:“诗三百,一言以蔽之,曰‘思无邪’。”

子曰:“道之以政,齐之以刑,民免而无耻;道之以德,齐之以礼,有耻且格。”

子曰:“吾十有五而志于学,三十而立,四十而不惑,五十而知天命,六十而耳顺,七十而从心所欲,不踰矩。”

孟懿子问孝。子曰:“无违。”樊迟御,子告之曰:“孟孙问孝于我,我对曰‘无违’。”樊迟曰:“何谓也?”子曰:“生,事之以礼;死,葬之以礼,祭之以礼。”

孟武伯问孝。子曰:“父母唯其疾之忧。”

子游问孝。子曰:“今之孝者,是谓能养。至于犬马,皆能有养;不敬,何以别乎?”

子夏问孝。子曰:“色难。有事弟子服其劳,有酒食先生馔,曾是以为孝乎?”

子曰:“吾与回言终日,不违如愚。退而省其私,亦足以发。回也,不愚。”

子曰:“视其所以,观其所由,察其所安。人焉廋哉?人焉廋哉?”

子曰:“温故而知新,可以为师矣。”

子曰:“君子不器。”

子贡问君子。子曰:“先行其言,而后从之。”

子曰:“君子周而不比,小人比而不周。”

子曰:“学而不思则罔,思而不学则殆。”

子曰:“攻乎异端,斯害也已!”

子曰:“由!诲女知之乎?知之为知之,不知为不知,是知也。”

子张学干禄。子曰:“多闻阙疑,慎言其余,则寡尤;多见阙殆,慎行其余,则寡悔。言寡尤,行寡悔,禄在其中矣。”

哀公问曰:“何为则民服?”孔子对曰:“举直错诸枉,则民服;举枉错诸直,则民不服。”

季康子问:“使民敬、忠以劝,如之何?”子曰:“临之以庄则敬,孝慈则忠,举善而教不能,则劝。”

或谓孔子曰:“子奚不为政?”子曰:“书云:‘孝乎惟孝、友于兄弟,施于有政。’是亦为政,奚其为为政?”

子曰:“人而无信,不知其可也。大车无輗,小车无軏,其何以行之哉?”

子张问:“十世可知也?”子曰:“殷因于夏礼,所损益,可知也;周因于殷礼,所损益,可知也;其或继周者,虽百世可知也。”

子曰:“非其鬼而祭之,谄也。见义不为,无勇也。”

\chapter{八佾第三}\label{ch3}
%卷三 八佾第三

孔子谓季氏:“八佾舞于庭,是可忍也,孰不可忍也?”

三家者以雍彻。子曰:“‘相维辟公,天子穆穆’,奚取于三家之堂?”

子曰:“人而不仁,如礼何?人而不仁,如乐何?”

林放问礼之本。子曰:“大哉问!礼,与其奢也,宁俭;丧,与其易也,宁戚。”

子曰:“夷狄之有君,不如诸夏之亡也。”

季氏旅于泰山。子谓冉有曰:“女弗能救与?”对曰:“不能。”子曰:“呜呼!曾谓泰山,不如林放乎?”

子曰:“君子无所争,必也射乎!揖让而升,下而饮,其争也君子。”

子夏问曰:“‘巧笑倩兮,美目盼兮,素以为绚兮。’何谓也?”子曰:“绘事后素。”曰:“礼后乎?”子曰:“起予者商也!始可与言诗已矣。”

子曰:“夏礼,吾能言之,杞不足征也;殷礼,吾能言之,宋不足征也。文献不足故也,足则吾能征之矣。”

子曰:“禘自既灌而往者,吾不欲观之矣。”

或问禘之说。子曰:“不知也。知其说者之于天下也,其如示诸斯乎!”指其掌。

祭如在,祭神如神在。子曰:“吾不与祭,如不祭。”

王孙贾问曰:“与其媚于奥,宁媚于灶,何谓也?”子曰:“不然,获罪于天,无所祷也。”

子曰:“周监于二代,郁郁乎文哉!吾从周。”

子入大庙,每事问。或曰:“孰谓鄹人之子知礼乎?入大庙,每事问。”子闻之曰:“是礼也。”

子曰:“射不主皮,为力不同科,古之道也。”

子贡欲去告朔之饩羊。子曰:“赐也,尔爱其羊,我爱其礼。”

子曰:“事君尽礼,人以为谄也。”

定公问:“君使臣,臣事君,如之何?”孔子对曰:“君使臣以礼,臣事君以忠。”

子曰:“关雎,乐而不淫,哀而不伤。”

哀公问社于宰我。宰我对曰:“夏后氏以松,殷人以柏,周人以栗,曰使民战栗。”子闻之曰:“成事不说,遂事不谏,既往不咎。”

子曰:“管仲之器小哉!”或曰:“管仲俭乎?”曰:“管氏有三归,官事不摄,焉得俭?”“然则管仲知礼乎?”曰:“邦君树塞门,管氏亦树塞门;邦君为两君之好,有反坫,管氏亦有反坫。管氏而知礼,孰不知礼?”

子语鲁大师乐。曰:“乐其可知也:始作,翕如也;从之,纯如也,皦如也,绎如也,以成。”

仪封人请见。曰:“君子之至于斯也,吾未尝不得见也。”从者见之。出曰:“二三子,何患于丧乎?天下之无道也久矣,天将以夫子为木铎。”

子谓韶,“尽美矣,又尽善也。”谓武,“尽美矣,未尽善也”。

子曰:“居上不宽,为礼不敬,临丧不哀,吾何以观之哉?”

\chapter{里仁第四}\label{ch4}
%卷四 里仁第四

子曰:“里仁为美。择不处仁,焉得知?”

子曰:“不仁者不可以久处约,不可以长处乐。仁者安仁,知者利仁。”

子曰:“唯仁者能好人,能恶人。”

子曰:“苟志于仁矣,无恶也。”

子曰:“富与贵是人之所欲也,不以其道得之,不处也;贫与贱是人之所恶也,不以其道得之,不去也。君子去仁,恶乎成名?君子无终食之间违仁,造次必于是,颠沛必于是。”

子曰:“我未见好仁者,恶不仁者。好仁者,无以尚之;恶不仁者,其为仁矣,不使不仁者加乎其身。有能一日用其力于仁矣乎?我未见力不足者。盖有之矣,我未之见也。”

子曰:“人之过也,各于其党。观过,斯知仁矣。”

子曰:“朝闻道,夕死可矣。”

子曰:“士志于道,而耻恶衣恶食者,未足与议也。”

子曰:“君子之于天下也,无适也,无莫也,义之与比。”

子曰:“君子怀德,小人怀土;君子怀刑,小人怀惠。”

子曰:“放于利而行,多怨。”

子曰:“能以礼让为国乎?何有?不能以礼让为国,如礼何?”

子曰:“不患无位,患所以立;不患莫己知,求为可知也。”

子曰:“参乎!吾道一以贯之。”曾子曰:“唯。”子出。门人问曰:“何谓也?”曾子曰:“夫子之道,忠恕而已矣。”

子曰:“君子喻于义,小人喻于利。”

子曰:“见贤思齐焉,见不贤而内自省也。”

子曰:“事父母几谏。见志不从,又敬不违,劳而不怨。”

子曰:“父母在,不远游。游必有方。”

子曰:“三年无改于父之道,可谓孝矣。”

子曰:“父母之年,不可不知也。一则以喜,一则以惧。”

子曰:“古者言之不出,耻躬之不逮也。”

子曰:“以约失之者,鲜矣。”

子曰:“君子欲讷于言,而敏于行。”

子曰:“德不孤,必有邻。”

子游曰:“事君数,斯辱矣,朋友数,斯疏矣。” 

\chapter{公冶长第五}\label{ch5}
%卷五 公冶长第五

子谓公冶长,“可妻也。虽在缧绁之中,非其罪也”。以其子妻之。

子谓南容,“邦有道,不废;邦无道,免于刑戮”。以其兄之子妻之。

子谓子贱,“君子哉若人!鲁无君子者,斯焉取斯?”

子贡问曰:“赐也何如?”子曰:“女器也。”曰:“何器也?”曰:“瑚琏也。”

或曰:“雍也,仁而不佞。”子曰:“焉用佞?御人以口给,屡憎于人。不知其仁,焉用佞?”

子使漆雕开仕。对曰:“吾斯之未能信。”子说。

子曰:“道不行,乘桴浮于海。从我者其由与?”子路闻之喜。子曰:“由也好勇过我,无所取材。”

孟武伯问:“子路仁乎?”子曰:“不知也。”又问。子曰:“由也,千乘之国,可使治其赋也,不知其仁也。”“求也何如?”子曰:“求也,千室之邑,百乘之家,可使为之宰也,不知其仁也。”“赤也何如?”子曰:“赤也,束带立于朝,可使与宾客言也,不知其仁也。”

子谓子贡曰:“女与回也孰愈?”对曰:“赐也何敢望回。回也闻一以知十,赐也闻一以知二。”子曰:“弗如也!吾与女弗如也。”

宰予昼寝。子曰:“朽木不可雕也,粪土之墙不可杇也,于予与何诛。”子曰:“始吾于人也,听其言而信其行;今吾于人也,听其言而观其行。于予与改是。”

子曰:“吾未见刚者。”或对曰:“申枨。”子曰:“枨也欲,焉得刚?”

子贡曰:“我不欲人之加诸我也,吾亦欲无加诸人。”子曰:“赐也,非尔所及也。”

子贡曰:“夫子之文章,可得而闻也;夫子之言性与天道,不可得而闻也。”

子路有闻,未之能行,唯恐有闻。

子贡问曰:“孔文子何以谓之文也?”子曰:“敏而好学,不耻下问,是以谓之文也。”

子谓子产,“有君子之道四焉:其行己也恭,其事上也敬,其养民也惠,其使民也义。”

子曰:“晏平仲善与人交,久而敬之。”

子曰:“臧文仲居蔡,山节藻梲,何如其知也?”

子张问曰:“令尹子文三仕为令尹,无喜色;三已之,无愠色。旧令尹之政,必以告新令尹。何如?”子曰:“忠矣。”曰:“仁矣乎?”曰:“未知,焉得仁?”“崔子弒齐君,陈文子有马十乘,弃而违之。至于他邦,则曰:‘犹吾大夫崔子也。’违之。之一邦,则又曰:‘犹吾大夫崔子也。’违之。何如?”子曰:“清矣。”曰:“仁矣乎?”曰:“未知。焉得仁?”

季文子三思而后行。子闻之,曰:“再,斯可矣。”

子曰:“宁武子邦有道则知,邦无道则愚。其知可及也,其愚不可及也。”

子在陈曰:“归与!归与!吾党之小子狂简,斐然成章,不知所以裁之。”

子曰:“伯夷、叔齐不念旧恶,怨是用希。”

子曰:“孰谓微生高直?或乞醯焉,乞诸其邻而与之。”

子曰:“巧言、令色、足恭,左丘明耻之,丘亦耻之。匿怨而友其人,左丘明耻之,丘亦耻之。”

颜渊、季路侍。子曰:“盍各言尔志?”子路曰:“愿车马、衣轻裘,与朋友共。敝之而无憾。”颜渊曰:“愿无伐善,无施劳。”子路曰:“愿闻子之志。”子曰:“老者安之,朋友信之,少者怀之。”

子曰:“已矣乎!吾未见能见其过而内自讼者也。”

子曰:“十室之邑,必有忠信如丘者焉,不如丘之好学也。” 

\chapter{雍也第六}\label{ch6}
%卷六 雍也第六

子曰:“雍也可使南面。”仲弓问子桑伯子,子曰:“可也简。”仲弓曰:“居敬而行简,以临其民,不亦可乎?居简而行简,无乃大简乎?”子曰:“雍之言然。”

哀公问:“弟子孰为好学?”孔子对曰:“有颜回者好学,不迁怒,不贰过。不幸短命死矣!今也则亡,未闻好学者也。”

子华使于齐,冉子为其母请粟。子曰:“与之釜。”请益。曰:“与之庾。”冉子与之粟五秉。子曰:“赤之适齐也,乘肥马,衣轻裘。吾闻之也,君子周急不继富。”原思为之宰,与之粟九百,辞。子曰:“毋!以与尔邻里乡党乎!”

子谓仲弓曰:“犁牛之子骍且角,虽欲勿用,山川其舍诸?”

子曰:“回也,其心三月不违仁,其余则日月至焉而已矣。”

季康子问:“仲由可使从政也与?”子曰:“由也果,于从政乎何有?”曰:“赐也,可使从政也与?”曰:“赐也达,于从政乎何有?”曰:“求也,可使从政也与?”曰:“求也艺,于从政乎何有?”

季氏使闵子骞为费宰。闵子骞曰:“善为我辞焉。如有复我者,则吾必在汶上矣。”

伯牛有疾,子问之,自牖执其手,曰:“亡之,命矣夫!斯人也而有斯疾也!斯人也而有斯疾也!”

子曰:“贤哉,回也!一箪食,一瓢饮,在陋巷。人不堪其忧,回也不改其乐。贤哉,回也!”

冉求曰:“非不说子之道,力不足也。”子曰:“力不足者,中道而废。今女画。”

子谓子夏曰:“女为君子儒,无为小人儒。”

子游为武城宰。子曰:“女得人焉尔乎?”曰:“有澹台灭明者,行不由径。非公事,未尝至于偃之室也。”

子曰:“孟之反不伐,奔而殿。将入门,策其马,曰:‘非敢后也,马不进也。’”

子曰:“不有祝鮀之佞而有宋朝之美,难乎免于今之世矣!”

子曰:“谁能出不由户?何莫由斯道也?”

子曰:“质胜文则野,文胜质则史。文质彬彬,然后君子。”

子曰:“人之生也直,罔之生也幸而免。”

子曰:“知之者不如好之者,好之者不如乐之者。”

子曰:“中人以上,可以语上也;中人以下,不可以语上也。”

樊迟问知。子曰:“务民之义,敬鬼神而远之,可谓知矣。”问仁。曰:“仁者先难而后获,可谓仁矣。”

子曰:“知者乐水,仁者乐山;知者动,仁者静;知者乐,仁者寿。”

子曰:“齐一变,至于鲁;鲁一变,至于道。”

子曰:“觚不觚,觚哉!觚哉!”

宰我问曰:“仁者,虽告之曰:‘井有仁焉。’其从之也?”子曰:“何为其然也?君子可逝也,不可陷也;可欺也,不可罔也。”

子曰:“君子博学于文,约之以礼,亦可以弗畔矣夫!”

子见南子,子路不说。夫子矢之曰:“予所否者,天厌之!天厌之!”

子曰:“中庸之为德也,其至矣乎!民鲜久矣。”

子贡曰:“如有博施于民而能济众,何如?可谓仁乎?”子曰:“何事于仁,必也圣乎!尧舜其犹病诸!夫仁者,己欲立而立人,己欲达而达人。能近取譬,可谓仁之方也已。” 

\chapter{述而第七}\label{ch7}
%卷七 述而第七

子曰:“述而不作,信而好古,窃比于我老彭。”

子曰:“默而识之,学而不厌,诲人不倦,何有于我哉?”

子曰:“德之不修,学之不讲,闻义不能徙,不善不能改,是吾忧也。”

子之燕居,申申如也,夭夭如也。

子曰:“甚矣吾衰也!久矣吾不复梦见周公。”

子曰:“志于道,据于德,依于仁,游于艺。”

子曰:“自行束修以上,吾未尝无诲焉。”

子曰:“不愤不启,不悱不发,举一隅不以三隅反,则不复也。”

子食于有丧者之侧,未尝饱也。于是日哭,则不歌。

子谓颜渊曰:“用之则行,舍之则藏,唯我与尔有是夫!”子路曰:“子行三军,则谁与?”子曰:“暴虎冯河,死而无悔者,吾不与也。必也临事而惧,好谋而成者也。”

子曰:“富而可求也,虽执鞭之士,吾亦为之。如不可求,从吾所好。”

子之所慎:齐,战,疾。

子在齐闻韶,三月不知肉味。曰:“不图为乐之至于斯也!”

冉有曰:“夫子为卫君乎?”子贡曰:“诺。吾将问之。”入,曰:“伯夷、叔齐何人也?”曰:“古之贤人也。”曰:“怨乎?”曰:“求仁而得仁,又何怨。”出,曰:“夫子不为也。”

子曰:“饭疏食饮水,曲肱而枕之,乐亦在其中矣。不义而富且贵,于我如浮云。”

子曰:“加我数年,五十以学易,可以无大过矣。”

子所雅言,诗、书、执礼,皆雅言也。

叶公问孔子于子路,子路不对。子曰:“女奚不曰,其为人也,发愤忘食,乐以忘忧,不知老之将至云尔。”

子曰:“我非生而知之者,好古,敏以求之者也。”

子不语怪,力,乱,神。

子曰:“三人行,必有我师焉。择其善者而从之,其不善者而改之。”

子曰:“天生德于予,桓魋其如予何?”

子曰:“二三子以我为隐乎?吾无隐乎尔。吾无行而不与二三子者,是丘也。”

子以四教:文,行,忠,信。

子曰:“圣人,吾不得而见之矣;得见君子者,斯可矣。”子曰:“

善人,吾不得而见之矣;得见有恒者,斯可矣。亡而为有,虚而为盈,约而为泰,难乎有恒矣。”

子钓而不纲,弋不射宿。

子曰:“盖有不知而作之者,我无是也。多闻择其善者而从之,多见而识之,知之次也。”

互乡难与言,童子见,门人惑。子曰:“与其进也,不与其退也,唯何甚!人洁己以进,与其洁也,不保其往也。”

子曰:“仁远乎哉?我欲仁,斯仁至矣。”

陈司败问昭公知礼乎?孔子曰:“知礼。”孔子退,揖巫马期而进之,曰:“吾闻君子不党,君子亦党乎?君取于吴为同姓,谓之吴孟子。君而知礼,孰不知礼?”巫马期以告。子曰:“丘也幸,苟有过,人必知之。”

子与人歌而善,必使反之,而后和之。

子曰:“文,莫吾犹人也。躬行君子,则吾未之有得。”

子曰:“若圣与仁,则吾岂敢?抑为之不厌,诲人不倦,则可谓云尔已矣。”公西华曰:“正唯弟子不能学也。”

子疾病,子路请祷。子曰:“有诸?”子路对曰:“有之。诔曰:‘

祷尔于上下神只。’”子曰:“丘之祷久矣。”

子曰:“奢则不孙,俭则固。与其不孙也,宁固。”

子曰:“君子坦荡荡,小人长戚戚。”

子温而厉,威而不猛,恭而安。 

\chapter{泰伯第八}\label{ch8}
%卷八 泰伯第八

子曰:“泰伯,其可谓至德也已矣!三以天下让,民无得而称焉。”

子曰:“恭而无礼则劳,慎而无礼则葸,勇而无礼则乱,直而无礼则绞。君子笃于亲,则民兴于仁;故旧不遗,则民不偷。”

曾子有疾,召门弟子曰:“启予足!启予手!诗云‘战战兢兢,如临深渊,如履薄冰。’而今而后,吾知免夫!小子!”

曾子有疾,孟敬子问之。曾子言曰:“鸟之将死,其鸣也哀;人之将死,其言也善。君子所贵乎道者三:动容貌,斯远暴慢矣;正颜色,斯近信矣;出辞气,斯远鄙倍矣。笾豆之事,则有司存。”

曾子曰:“以能问于不能,以多问于寡;有若无,实若虚,犯而不校,昔者吾友尝从事于斯矣。”

曾子曰:“可以托六尺之孤,可以寄百里之命,临大节而不可夺也。君子人与?君子人也。”

曾子曰:“士不可以不弘毅,任重而道远。仁以为己任,不亦重乎?死而后已,不亦远乎?”

子曰:“兴于诗,立于礼。成于乐。”

子曰:“民可使由之,不可使知之。”

子曰:“好勇疾贫,乱也。人而不仁,疾之已甚,乱也。”

子曰:“如有周公之才之美,使骄且吝,其余不足观也已。”

子曰:“三年学,不至于谷,不易得也。”

子曰:“笃信好学,守死善道。危邦不入,乱邦不居。天下有道则见,无道则隐。邦有道,贫且贱焉,耻也;邦无道,富且贵焉,耻也。”

子曰:“不在其位,不谋其政。”

子曰:“师挚之始,关雎之乱,洋洋乎!盈耳哉。”

子曰:“狂而不直,侗而不愿,悾悾而不信,吾不知之矣。”

子曰:“学如不及,犹恐失之。”

子曰:“巍巍乎!舜禹之有天下也,而不与焉。”

子曰:“大哉,尧之为君也!巍巍乎!唯天为大,唯尧则之。荡荡乎!民无能名焉。巍巍乎!其有成功也;焕乎,其有文章!”

舜有臣五人而天下治。武王曰:“予有乱臣十人。”孔子曰:“才难,不其然乎?唐虞之际,于斯为盛。有妇人焉,九人而已。三分天下有其二,以服事殷。周之德,其可谓至德也已矣。”

子曰:“禹,吾无间然矣。菲饮食,而致孝乎鬼神;恶衣服,而致美乎黻冕;卑宫室,而尽力乎沟洫。禹,吾无间然矣。” 

\chapter{子罕第九}\label{ch9}
%卷九 子罕第九

子罕言利,与命,与仁。

达巷党人曰:“大哉孔子!博学而无所成名。”子闻之,谓门弟子曰:“吾何执?执御乎?执射乎?吾执御矣。”

子曰:“麻冕,礼也;今也纯,俭。吾从众。拜下,礼也;今拜乎上,泰也。虽违众,吾从下。”

子绝四:毋意,毋必,毋固,毋我。

子畏于匡。曰:“文王既没,文不在兹乎?天之将丧斯文也,后死者不得与于斯文也;天之未丧斯文也,匡人其如予何?”

大宰问于子贡曰:“夫子圣者与?何其多能也?”子贡曰:“固天纵之将圣,又多能也。”子闻之,曰:“大宰知我乎!吾少也贱,故多能鄙事。君子多乎哉?不多也。”牢曰:“子云,‘吾不试,故艺’。”

子曰:“吾有知乎哉?无知也。有鄙夫问于我,空空如也,我叩其两端而竭焉。”

子曰:“凤鸟不至,河不出图,吾已矣夫!”

子见齐衰者、冕衣裳者与瞽者,见之,虽少必作;过之,必趋。

颜渊喟然叹曰:“仰之弥高,钻之弥坚;瞻之在前,忽焉在后。夫子循循然善诱人,博我以文,约我以礼。欲罢不能,既竭吾才,如有所立卓尔。虽欲从之,末由也已。”

子疾病,子路使门人为臣。病闲,曰:“久矣哉!由之行诈也,无臣而为有臣。吾谁欺?欺天乎?且予与其死于臣之手也,无宁死于二三子之手乎?且予纵不得大葬,予死于道路乎?”

子贡曰:“有美玉于斯,韫匮而藏诸?求善贾而沽诸?”子曰:“沽之哉!沽之哉!我待贾者也。”

子欲居九夷。或曰:“陋,如之何!”子曰:“君子居之,何陋之有?”

子曰:“吾自卫反鲁,然后乐正,雅颂各得其所。”

子曰:“出则事公卿,入则事父兄,丧事不敢不勉,不为酒困,何有于我哉?”

子在川上,曰:“逝者如斯夫!不舍昼夜。”

子曰:“吾未见好德如好色者也。”

子曰:“譬如为山,未成一篑,止,吾止也;譬如平地,虽覆一篑,进,吾往也。”

子曰:“语之而不惰者,其回也与!”

子谓颜渊,曰:“惜乎!吾见其进也,未见其止也。”

子曰;“苗而不秀者有矣夫!秀而不实者有矣夫!”

子曰:“后生可畏,焉知来者之不如今也?四十、五十而无闻焉,斯亦不足畏也已。”

子曰:“法语之言,能无从乎?改之为贵。巽与之言,能无说乎?绎之为贵。说而不绎,从而不改,吾末如之何也已矣。”

子曰:“主忠信,毋友不如己者,过则勿惮改。”

子曰:“三军可夺帅也,匹夫不可夺志也。”

子曰:“衣敝缊袍,与衣狐貉者立,而不耻者,其由也与?‘不忮不求,何用不臧?’”子路终身诵之。子曰:“是道也,何足以臧?”

子曰:“岁寒,然后知松柏之后雕也。”

子曰:“知者不惑,仁者不忧,勇者不惧。”

子曰:“可与共学,未可与适道;可与适道,未可与立;可与立,未可与权。”

“唐棣之华,偏其反而。岂不尔思?室是远而。”子曰:“未之思也,夫何远之有?” 

\chapter{乡党第十}\label{ch10}
%卷十 乡党第十

孔子于乡党,恂恂如也,似不能言者。其在宗庙朝廷,便便言,唯谨尔。 

朝,与下大夫言,侃侃如也;与上大夫言,誾誾如也。君在,踧踖如也。与与如也。 

君召使摈,色勃如也,足躩如也。揖所与立,左右手。衣前后,襜如也。趋进,翼如也。宾退,必复命曰:“宾不顾矣。” 

入公门,鞠躬如也,如不容。立不中门,行不履阈。过位,色勃如也,足躩如也,其言似不足者。摄齐升堂,鞠躬如也,屏气似不息者。出,降一等,逞颜色,怡怡如也。没阶趋,翼如也。复其位,踧踖如也。 

执圭,鞠躬如也,如不胜。上如揖,下如授。勃如战色,足缩缩,如有循。享礼,有容色。私觌,愉愉如也。 

君子不以绀緅饰。红紫不以为亵服。当暑,袗絺绤,必表而出之。缁衣羔裘,素衣麑裘,黄衣狐裘。亵裘长。短右袂。狐貉之厚以居。去丧,无所不佩。非帷裳,必杀之。羔裘玄冠不以吊。吉月,必朝服而朝。 

齐,必有明衣,布。必有寝衣,长一身有半。齐,必变食,居必迁坐。 

食不厌精,脍不厌细。食饐而餲,鱼馁而肉败,不食。色恶,不食。臭恶,不食。失饪,不食。不时,不食。割不正,不食。不得其酱,不食。肉虽多,不使胜食气。惟酒无量,不及乱。沽酒市脯不食。不撤姜食。不多食。祭于公,不宿肉。祭肉不出三日。出三日,不食之矣。食不语,寝不言。虽疏食菜羹,瓜祭,必齐如也。 

席不正,不坐。 

乡人饮酒,杖者出,斯出矣。乡人傩,朝服而立于阼阶。 

问人于他邦,再拜而送之。康子馈药,拜而受之。曰:“丘未达,不敢尝。” 

厩焚。子退朝,曰:“伤人乎?”不问马。 

君赐食,必正席先尝之;君赐腥,必熟而荐之;君赐生,必畜之。侍食于君,君祭,先饭。疾,君视之,东首,加朝服,拖绅。君命召,不俟驾行矣。 

入太庙,每事问。 

朋友死,无所归。曰:“于我殡。”朋友之馈,虽车马,非祭肉,不拜。 

寝不尸,居不容。见齐衰者,虽狎,必变。见冕者与瞽者,虽亵,必以貌。凶服者式之。式负版者。有盛馔,必变色而作。迅雷风烈,必变。 

升车,必正立执绥。车中,不内顾,不疾言,不亲指。 

色斯举矣,翔而后集。曰:“山梁雌雉,时哉!时哉!”子路共之,三嗅而作。 

\chapter{先进第十一}\label{ch11}
%卷十一 先进第十一

子曰:“先进于礼乐,野人也;后进于礼乐,君子也。如用之,则吾从先进。” 

子曰:“从我于陈、蔡者,皆不及门也。”德行:颜渊,闵子骞,冉伯牛,仲弓。言语:宰我,子贡。政事:冉有,季路。文学:子游,子夏。 

子曰:“回也非助我者也,于吾言无所不说。” 

子曰:“孝哉闵子骞!人不间于其父母昆弟之言。” 

南容三复白圭,孔子以其兄之子妻之。 

季康子问:“弟子孰为好学?”孔子对曰:“有颜回者好学,不幸短命死矣!今也则亡。” 

颜渊死,颜路请子之车以为之椁。子曰:“才不才,亦各言其子也。鲤也死,有棺而无椁。吾不徒行以为之椁。以吾从大夫之后,不可徒行也。” 

颜渊死。子曰:“噫!天丧予!天丧予!” 

颜渊死,子哭之恸。从者曰:“子恸矣。”曰:“有恸乎?非夫人之为恸而谁为!” 

颜渊死,门人欲厚葬之,子曰:“不可。”门人厚葬之。子曰:“回也视予犹父也,予不得视犹子也。非我也,夫二三子也。” 

季路问事鬼神。子曰:“未能事人,焉能事鬼?”敢问死。曰:“未知生,焉知死?” 

闵子侍侧,誾誾如也;子路,行行如也;冉有、子贡,侃侃如也。子乐。“若由也,不得其死然。” 

鲁人为长府。闵子骞曰:“仍旧贯,如之何?何必改作?”子曰:“夫人不言,言必有中。” 

子曰:“由之瑟奚为于丘之门?”门人不敬子路。子曰:“由也升堂矣,未入于室也。” 

子贡问:“师与商也孰贤?”子曰:“师也过,商也不及。”曰:“然则师愈与?”子曰:“过犹不及。” 

季氏富于周公,而求也为之聚敛而附益之。子曰:“非吾徒也。小子鸣鼓而攻之,可也。” 

柴也愚,参也鲁,师也辟,由也喭。 

子曰:“回也其庶乎,屡空。赐不受命,而货殖焉,亿则屡中。” 

子张问善人之道。子曰:“不践迹,亦不入于室。” 

子曰:“论笃是与,君子者乎?色庄者乎?” 

子路问:“闻斯行诸?”子曰:“有父兄在,如之何其闻斯行之?”冉有问:“闻斯行诸?”子曰:“闻斯行之。”公西华曰:“由也问闻斯行诸,子曰‘有父兄在’;求也问闻斯行诸,子曰‘闻斯行之’。赤也惑,敢问。”子曰:“求也退,故进之;由也兼人,故退之。” 

子畏于匡,颜渊后。子曰:“吾以女为死矣。”曰:“子在,回何敢死?” 

季子然问:“仲由、冉求可谓大臣与?”子曰:“吾以子为异之问,曾由与求之问。所谓大臣者:以道事君,不可则止。今由与求也,可谓具臣矣。”曰:“然则从之者与?”子曰:“弒父与君,亦不从也。” 

子路使子羔为费宰。子曰:“贼夫人之子。”子路曰:“有民人焉,有社稷焉。何必读书,然后为学?”子曰:“是故恶夫佞者。” 

子路、曾皙、冉有、公西华侍坐。子曰:“以吾一日长乎尔,毋吾以也。居则曰:“不吾知也!’如或知尔,则何以哉?”子路率尔而对曰:“千乘之国,摄乎大国之间,加之以师旅,因之以饥馑;由也为之,比及三年,可使有勇,且知方也。”夫子哂之。“求!尔何如?”对曰:“方六七十,如五六十,求也为之,比及三年,可使足民。如其礼乐,以俟君子。”“赤!尔何如?”对曰:“非曰能之,愿学焉。宗庙之事,如会同,端章甫,愿为小相焉。”“点!尔何如?”鼓瑟希,铿尔,舍瑟而作。对曰:“异乎三子者之撰。”子曰:“何伤乎?亦各言其志也。”曰:“莫春者,春服既成。冠者五六人,童子六七人,浴乎沂,风乎舞雩,咏而归。”夫子喟然叹曰:“吾与点也!”三子者出,曾皙后。曾皙曰:“夫三子者之言何如?”子曰:“亦各言其志也已矣。”曰:“夫子何哂由也?”曰:“为国以礼,其言不让,是故哂之。”“唯求则非邦也与?”“安见方六七十如五六十而非邦也者?”“唯赤则非邦也与?”“宗庙会同,非诸侯而何?赤也为之小,孰能为之大?” 


\chapter{颜渊第十二}\label{ch12}
%卷十二 颜渊第十二

颜渊问仁。子曰:“克己复礼为仁。一日克己复礼,天下归仁焉。为仁由己,而由人乎哉?”颜渊曰:“请问其目。”子曰:“非礼勿视,非礼勿听,非礼勿言,非礼勿动。”颜渊曰:“回虽不敏,请事斯语矣。”

仲弓问仁。子曰:“出门如见大宾,使民如承大祭。己所不欲,勿施于人。在邦无怨,在家无怨。”仲弓曰:“雍虽不敏,请事斯语矣。”

司马牛问仁。子曰:“仁者其言也讱。”曰:“其言也讱,斯谓之仁已乎?”子曰:“为之难,言之得无讱乎?”

司马牛问君子。子曰:“君子不忧不惧。”曰:“不忧不惧,斯谓之君子已乎?”子曰:“内省不疚,夫何忧何惧?”

司马牛忧曰:“人皆有兄弟,我独亡。”子夏曰:“商闻之矣:死生有命,富贵在天。君子敬而无失,与人恭而有礼。四海之内,皆兄弟也。君子何患乎无兄弟也?”

子张问明。子曰:“浸润之谮,肤受之愬,不行焉。可谓明也已矣。浸润之谮肤受之愬不行焉,可谓远也已矣。”

子贡问政。子曰:“足食。足兵。民信之矣。”子贡曰:“必不得已而去,于斯三者何先?”曰:“去兵。”子贡曰:“必不得已而去,于斯二者何先?”曰:“去食。自古皆有死,民无信不立。”

棘子成曰:“君子质而已矣,何以文为?”子贡曰:“惜乎!夫子之说,君子也。驷不及舌。文犹质也,质犹文也。虎豹之(革享),犹犬羊之(革享)。”

哀公问于有若曰:“年饥,用不足,如之何?”有若对曰:“盍彻乎?”曰:“二,吾犹不足,如之何其彻也?”对曰:“百姓足,君孰与不足?百姓不足,君孰与足?”

子张问崇德、辨惑。子曰:“主忠信,徙义,崇德也。爱之欲其生,恶之欲其死。既欲其生,又欲其死,是惑也。”

齐景公问政于孔子。孔子对曰:“君君,臣臣,父父,子子。”公曰:“善哉!信如君不君,臣不臣,父不父,子不子,虽有粟,吾得而食诸?”

子曰:“片言可以折狱者,其由也与?”子路无宿诺。

子曰:“听讼,吾犹人也,必也使无讼乎!”

子张问政。子曰:“居之无倦,行之以忠。”

子曰:“君子博学于文,约之以礼,亦可以弗畔矣夫!”

子曰:“君子成人之美,不成人之恶。小人反是。”

季康子问政于孔子。孔子对曰:“政者,正也。子帅以正,孰敢不正?”

季康子患盗,问于孔子。孔子对曰:“苟子之不欲,虽赏之不窃。”

季康子问政于孔子曰:“如杀无道,以就有道,何如?”孔子对曰:“子为政,焉用杀?子欲善,而民善矣。君子之德风,小人之德草。草上之风,必偃。”

子张问:“士何如斯可谓之达矣?”子曰:“何哉,尔所谓达者?”子张对曰:“在邦必闻,在家必闻。”子曰:“是闻也,非达也。夫达也者,质直而好义,察言而观色,虑以下人。在邦必达,在家必达。夫闻也者,色取仁而行违,居之不疑。在邦必闻,在家必闻。”

樊迟从游于舞雩之下,曰:“敢问崇德、修慝、辨惑。”子曰:“善哉问!先事后得,非崇德与?攻其恶,无攻人之恶,非修慝与?一朝之忿,忘其身,以及其亲,非惑与?”

樊迟问仁。子曰:“爱人。”问知。子曰:“知人。”樊迟未达。子曰:“举直错诸枉,能使枉者直。”樊迟退,见子夏。曰:“乡也吾见于夫子而问知,子曰,‘举直错诸枉,能使枉者直’,何谓也?”子夏曰:“富哉言乎!舜有天下,选于众,举皋陶,不仁者远矣。汤有天下,选于众,举伊尹,不仁者远矣。”

子贡问友。子曰:“忠告而善道之,不可则止,无自辱焉。”

曾子曰:“君子以文会友,以友辅仁。” 

\chapter{子路第十三}\label{ch13}
%卷十三 子路第十三

子路问政。子曰:“先之,劳之。”请益。曰:“无倦。”

仲弓为季氏宰,问政。子曰:“先有司,赦小过,举贤才。”曰:“焉知贤才而举之?”曰:“举尔所知。尔所不知,人其舍诸?”

子路曰:“卫君待子而为政,子将奚先?”子曰:“必也正名乎!”子路曰:“有是哉,子之迂也!奚其正?”子曰:“野哉由也!君子于其所不知,盖阙如也。名不正,则言不顺;言不顺,则事不成;事不成,则礼乐不兴;礼乐不兴,则刑罚不中;刑罚不中,则民无所措手足。故君子名之必可言也,言之必可行也。君子于其言,无所苟而已矣。”

樊迟请学稼,子曰:“吾不如老农。”请学为圃。曰:“吾不如老圃。”樊迟出。子曰:“小人哉,樊须也!上好礼,则民莫敢不敬;上好义,则民莫敢不服;上好信,则民莫敢不用情。夫如是,则四方之民襁负其子而至矣,焉用稼?”

子曰:“诵诗三百,授之以政,不达;使于四方,不能专对;虽多,亦奚以为?”

子曰:“其身正,不令而行;其不正,虽令不从。”

子曰:“鲁卫之政,兄弟也。”

子谓卫公子荆,“善居室。始有,曰:‘苟合矣。’少有,曰:‘苟完矣。’富有,曰:‘苟美矣。’”

子适卫,冉有仆。子曰:“庶矣哉!”冉有曰:“既庶矣。又何加焉?”曰:“富之。”曰:“既富矣,又何加焉?”曰:“教之。”

子曰:“苟有用我者。期月而已可也,三年有成。”

子曰:“善人为邦百年,亦可以胜残去杀矣。诚哉是言也!”

子曰:“如有王者,必世而后仁。”

子曰:“苟正其身矣,于从政乎何有?不能正其身,如正人何?”

冉子退朝。子曰:“何晏也?”对曰:“有政。”子曰:“其事也。如有政,虽不吾以,吾其与闻之。”

定公问:“一言而可以兴邦,有诸?”孔子对曰:“言不可以若是其几也。人之言曰:‘为君难,为臣不易。’如知为君之难也,不几乎一言而兴邦乎?”曰:“一言而丧邦,有诸?”孔子对曰:“言不可以若是其几也。人之言曰:‘予无乐乎为君,唯其言而莫予违也。’如其善而莫之违也,不亦善乎?如不善而莫之违也,不几乎一言而丧邦乎?”

叶公问政。子曰:“近者说,远者来。”

子夏为莒父宰,问政。子曰:“无欲速,无见小利。欲速,则不达;见小利,则大事不成。”

叶公语孔子曰:“吾党有直躬者,其父攘羊,而子证之。”孔子曰:“吾党之直者异于是。父为子隐,子为父隐,直在其中矣。”

樊迟问仁。子曰:“居处恭,执事敬,与人忠。虽之夷狄,不可弃也。”

子贡问曰:“何如斯可谓之士矣?”子曰:“行己有耻,使于四方,不辱君命,可谓士矣。”曰:“敢问其次。”曰:“宗族称孝焉,乡党称弟焉。”曰:“敢问其次。”曰:“言必信,行必果,硁硁然小人哉!抑亦可以为次矣。”曰:“今之从政者何如?”子曰:“噫!斗筲之人,何足算也。”

子曰:“不得中行而与之,必也狂狷乎!狂者进取,狷者有所不为也。”

子曰:“南人有言曰:‘人而无恒,不可以作巫医。’善夫!”“不恒其德,或承之羞。”子曰:“不占而已矣。”

子曰:“君子和而不同,小人同而不和。”

子贡问曰:“乡人皆好之,何如?”子曰:“未可也。”“乡人皆恶之,何如?”子曰:“未可也。不如乡人之善者好之,其不善者恶之。”

子曰:“君子易事而难说也:说之不以道,不说也;及其使人也,器之。小人难事而易说也:说之虽不以道,说也;及其使人也,求备焉。”

子曰:“君子泰而不骄,小人骄而不泰。”

子曰:“刚毅、木讷,近仁。”

子路问曰:“何如斯可谓之士矣?”子曰:“切切、偲偲、怡怡如也,可谓士矣。朋友切切、偲偲,兄弟怡怡。”

子曰:“善人教民七年,亦可以即戎矣。”

子曰:“以不教民战,是谓弃之。” 

\chapter{宪问第十四}\label{ch14}
%卷十四 宪问第十四

宪问耻。子曰:“邦有道,谷;邦无道,谷,耻也。”

“克、伐、怨、欲不行焉,可以为仁矣?”子曰:“可以为难矣,仁则吾不知也。”

子曰:“士而怀居,不足以为士矣。”

子曰:“邦有道,危言危行;邦无道,危行言孙。”

子曰:“有德者,必有言。有言者,不必有德。仁者,必有勇。勇者,不必有仁。”

南宫适问于孔子曰:“羿善射,奡荡舟,俱不得其死然;禹稷躬稼,而有天下。”夫子不答,南宫适出。子曰:“君子哉若人!尚德哉若人!”

子曰:“君子而不仁者有矣夫,未有小人而仁者也。”

子曰:“爱之,能勿劳乎?忠焉,能勿诲乎?”

子曰:“为命:裨谌草创之,世叔讨论之,行人子羽修饰之,东里子产润色之。”

或问子产。子曰:“惠人也。”问子西。曰:“彼哉!彼哉!”问管仲。曰:“人也。夺伯氏骈邑三百,饭疏食,没齿,无怨言。”

子曰:“贫而无怨难,富而无骄易。”

子曰:“孟公绰,为赵魏老则优,不可以为滕薛大夫。”

子路问成人。子曰:“若臧武仲之知,公绰之不欲,卞庄子之勇,冉求之艺,文之以礼乐,亦可以为成人矣。”曰:“今之成人者何必然?见利思义,见危授命,久要不忘平生之言,亦可以为成人矣。”

子问公叔文子于公明贾曰:“信乎夫子不言、不笑、不取乎?”公明贾对曰:“以告者过也。夫子时然后言,人不厌其言;乐然后笑,人不厌其笑;义然后取,人不厌其取。”子曰:“其然,岂其然乎?”

子曰:“臧武仲以防求为后于鲁,虽曰不要君,吾不信也。”

子曰:“晋文公谲而不正,齐桓公正而不谲。”

子路曰:“桓公杀公子纠,召忽死之,管仲不死。”曰:“未仁乎?”子曰:“桓公九合诸侯,不以兵车,管仲之力也。如其仁!如其仁!”

子贡曰:“管仲非仁者与?桓公杀公子纠,不能死,又相之。”子曰:“管仲相桓公,霸诸侯,一匡天下,民到于今受其赐。微管仲,吾其被发左衽矣。岂若匹夫匹妇之为谅也,自经于沟渎,而莫之知也。”

公叔文子之臣大夫僎,与文子同升诸公。子闻之曰:“可以为文矣。”

子言卫灵公之无道也,康子曰:“夫如是,奚而不丧?”孔子曰:“仲叔圉治宾客,祝鮀治宗庙,王孙贾治军旅。夫如是,奚其丧?”

子曰:“其言之不怍,则为之也难。”

陈成子弒简公。孔子沐浴而朝,告于哀公曰:“陈恒弒其君,请讨之。”公曰:“告夫三子!”孔子曰:“以吾从大夫之后,不敢不告也。君曰‘告夫三子’者。”之三子告,不可。孔子曰:“以吾从大夫之后,不敢不告也。”

子路问事君。子曰:“勿欺也,而犯之。”

子曰:“君子上达,小人下达。”

子曰:“古之学者为己,今之学者为人。”

蘧伯玉使人于孔子。孔子与之坐而问焉,曰:“夫子何为?”对曰:“夫子欲寡其过而未能也。”使者出。子曰:“使乎!使乎!”

子曰:“不在其位,不谋其政。”

曾子曰:“君子思不出其位。”

子曰:“君子耻其言而过其行。”

子曰:“君子道者三,我无能焉:仁者不忧,知者不惑,勇者不惧。”子贡曰:“夫子自道也。”

子贡方人。子曰:“赐也贤乎哉?夫我则不暇。”

子曰:“不患人之不己知,患其不能也。”

子曰:“不逆诈,不亿不信。抑亦先觉者,是贤乎!”

微生亩谓孔子曰:“丘何为是栖栖者与?无乃为佞乎?”孔子曰:“非敢为佞也,疾固也。”

子曰:“骥不称其力,称其德也。”

或曰:“以德报怨,何如?”子曰:“何以报德?以直报怨,以德报德。”

子曰:“莫我知也夫!”子贡曰:“何为其莫知子也?”子曰:“不怨天,不尤人。下学而上达。知我者,其天乎!”

公伯寮愬子路于季孙。子服景伯以告,曰:“夫子固有惑志于公伯寮,吾力犹能肆诸市朝。”子曰:“道之将行也与?命也。道之将废也与?命也。公伯寮其如命何!”

子曰:“贤者辟世,其次辟地,其次辟色,其次辟言。”

子曰:“作者七人矣。”

子路宿于石门。晨门曰:“奚自?”子路曰:“自孔氏。”曰:“是知其不可而为之者与?”

子击磬于卫。有荷蒉而过孔氏之门者,曰:“有心哉!击磬乎!”既而曰:“鄙哉!硁硁乎!莫己知也,斯己而已矣。深则厉,浅则揭。”子曰:“果哉!末之难矣。”

子张曰:“书云:‘高宗谅阴,三年不言。’何谓也?”子曰:“何必高宗,古之人皆然。君薨,百官总己以听于冢宰,三年。”

子曰:“上好礼,则民易使也。”

子路问君子。子曰:“修己以敬。”曰:“如斯而已乎?”曰:“修己以安人。”曰:“如斯而已乎?”曰:“修己以安百姓。修己以安百姓,尧舜其犹病诸!”

原壤夷俟。子曰:“幼而不孙弟,长而无述焉,老而不死,是为贼!”以杖叩其胫。

阙党童子将命。或问之曰:“益者与?”子曰:“吾见其居于位也,见其与先生并行也。非求益者也,欲速成者也。” 

\chapter{卫灵公第十五}\label{ch15}
%《论语·卫灵公》

卫灵公问陈于孔子。孔子对曰:“俎豆之事,则尝闻之矣;军旅之事,未之学也。”明日遂行。

在陈绝粮,从者病,莫能兴。子路愠见曰:“君子亦有穷乎?”子曰:“君子固穷,小人穷斯滥矣。”

子曰;“赐也,女以予为多学而识之者与?”对曰:“然,非与?”曰:“非也,予一以贯之。”

子曰:“由!知德者鲜矣。”

子曰:“无为而治者其舜也与?夫何为哉?恭己正南面而已矣。”

子张问行。子曰:“言忠信,行笃敬,虽蛮貊之邦,行矣。言不忠信,行不笃敬,虽州里,行乎哉?立则见其参于前也,在舆则见其倚于衡也,夫然后行。”子张书诸绅。

子曰:“直哉史鱼!邦有道,如矢;邦无道,如矢。君子哉蘧伯玉!邦有道,则仕;邦无道,则可卷而怀之。”

子曰:“可与言而不与之言,失人;不可与言而与之言,失言。知者不失人,亦不失言。”

子曰:“志士仁人,无求生以害仁,有杀身以成仁。”

子贡问为仁。子曰:“工欲善其事,必先利其器。居是邦也,事其大夫之贤者,友其士之仁者。”

颜渊问为邦。子曰:“行夏之时,乘殷之辂,服周之冕,乐则《韶》《舞》,放郑声,远佞人。郑声淫,佞人殆。”

子曰:“人无远虑,必有近忧。”

子曰:“已矣乎!吾未见好德如好色者也。”

子曰:“臧文仲其窃位者与!知柳下惠之贤而不与立也。”

子曰:“躬自厚而薄责于人,则远怨矣。”

子曰:“不曰‘如之何,如之何’者,吾末如之何也已矣。”

子曰:“群居终日,言不及义,好行小惠,难矣哉!”

子曰:“君子义以为质,礼以行之,孙以出之,信以成之。君子哉!”

子曰:“君子病无能焉,不病人之不己知也。”

子曰:“君子疾没世而名不称焉。”

子曰:“君子求诸己,小人求诸人。”

子曰:“君子矜而不争,群而不党。”

子曰:“君子不以言举人,不以人废言。”

子贡问曰:“有一言而可以终身行之者乎?”子曰:“其恕乎!己所不欲,勿施于人。”

子曰:“吾之于人也,谁毁谁誉?如有所誉者,其有所试矣。斯民也,三代之所以直道而行也。”

子曰:“吾犹及史之阙文也。有马者借人乘之,今亡也夫!”

子曰:“巧言乱德。小不忍,则乱大谋。”

子曰:“众恶之,必察焉;众好之,必察焉。”

子曰:“人能弘道,非道弘人。”

子曰:“过而不改,是谓过矣。”

子曰:“吾尝终日不食,终夜不寝,以思,无益,不如学也。”

子曰:“君子谋道不谋食。耕也,馁在其中矣;学也,禄在其中矣。君子忧道不忧贫。”

子曰:“知及之,仁不能守之;虽得之,必失之。知及之,仁能守之。不庄以涖之,则民不敬。知及之,仁能守之,庄以涖之,动之不以礼,未善也。”

子曰:“君子不可小知而可大受也,小人不可大受而可小知也。”

子曰:“民之于仁也,甚于水火。水火,吾见蹈而死者矣,未见蹈仁而死者也。”

子曰:“当仁,不让于师。”

子曰:“君子贞而不谅。”

子曰:“事君,敬其事而后其食。”

子曰:“有教无类。”

子曰:“道不同,不相为谋。”

子曰:“辞达而已矣。”

师冕见,及阶,子曰:“阶也。”及席,子曰:“席也。”皆坐,子告之曰:“某在斯,某在斯。”师冕出。子张问曰:“与师言之道与?”子曰:“然;固相师之道也。”


\chapter{季氏第十六}\label{ch16}
%《论语·季氏》

季氏将伐颛臾。冉有、季路见于孔子曰:“季氏有事于颛臾。”孔子曰:“求!无乃尔是过与?夫颛臾,昔者先王以为东蒙主,且在邦域之中矣,是社稷之臣也。何以伐为?”冉有曰:“夫子欲之,吾二臣者皆不欲也。”孔子曰:“求!周任有言曰:‘陈力就列,不能者止。’危而不持,颠而不扶,则将焉用彼相矣?且尔言过矣,虎兕出于柙,龟玉毁于椟中,是谁之过与?”冉有曰:“今夫颛臾,固而近于费。今不取,后世必为子孙忧。”孔子曰:“求!君子疾夫舍曰欲之而必为之辞。丘也闻有国有家者,不患寡而患不均,不患贫而患不安。盖均无贫,和无寡,安无倾。夫如是,故远人不服,则修文德以来之。既来之,则安之。今由与求也,相夫子,远人不服,而不能来也;邦分崩离析,而不能守也;而谋动干戈于邦内。吾恐季孙之忧,不在颛臾,而在萧墙之内也。”

孔子曰:“天下有道,则礼乐征伐自天子出;天下无道,则礼乐征伐自诸侯出。自诸侯出,盖十世希不失矣;自大夫出,五世希不失矣;陪臣执国命,三世希不失矣。天下有道,则政不在大夫。天下有道,则庶人不议。”

孔子曰:“禄之去公室五世矣,政逮于大夫四世矣,故夫三桓之子孙微矣。”

孔子曰:“益者三友,损者三友。友直,友谅,友多闻,益矣。友便辟,友善柔,友便佞,损矣。”

孔子曰:“益者三乐,损者三乐。乐节礼乐,乐道人之善,乐多贤友,益矣。乐骄乐,乐佚游,乐宴乐,损矣。”

孔子曰:“侍于君子有三愆:言未及之而言谓之躁,言及之而不言谓之隐,未见颜色而言谓之瞽。”

孔子曰:“君子有三戒:少之时,血气未定,戒之在色;及其壮也,血气方刚,戒之在斗;及其老也,血气既衰,戒之在得。”

孔子曰:“君子有三畏:畏天命,畏大人,畏圣人之言。小人不知天命而不畏也,狎大人,侮圣人之言。”

孔子曰:“生而知之者上也,学而知之者次也;困而学之,又其次也;困而不学,民斯为下矣。”

孔子曰:“君子有九思:视思明,听思聪,色思温,貌思恭,言思忠,事思敬,疑思问,忿思难,见得思义。”

孔子曰:“见善如不及,见不善如探汤。吾见其人矣,吾闻其语矣。隐居以求其志,行义以达其道。吾闻其语矣,未见其人也。”

齐景公有马千驷,死之日,民无德而称焉。伯夷叔齐饿于首阳之下,民到于今称之。其斯之谓与?

陈亢问于伯鱼曰:“子亦有异闻乎?”对曰:“未也。尝独立,鲤趋而过庭。曰:‘学诗乎?’对曰:‘未也。’‘不学诗,无以言。’鲤退而学诗。他日,又独立,鲤趋而过庭。曰:‘学礼乎?’对曰:‘未也。’‘不学礼,无以立。’鲤退而学礼。闻斯二者。”陈亢退而喜曰:“问一得三,闻诗,闻礼,又闻君子之远其子也。”

邦君之妻,君称之曰夫人,夫人自称曰小童;邦人称之曰君夫人,称诸异邦曰寡小君;异邦人称之亦曰君夫人。


\chapter{阳货第十七}\label{ch17}
%《论语·阳货》

阳货欲见孔子,孔子不见,归孔子豚。孔子时其亡也,而往拜之。遇诸涂。谓孔子曰:“来!予与尔言。”曰:“怀其宝而迷其邦,可谓仁乎?”曰:“不可。——好从事而亟失时,可谓知乎?”曰:“不可。——日月逝矣,岁不我与。”孔子曰:“诺;吾将仕矣。”

子曰:“性相近也,习相远也。”

子曰:“唯上智与下愚不移。”

子之武城,闻弦歌之声。夫子莞尔而笑,曰:“割鸡焉用牛刀?”子游对曰:“昔者偃也闻诸夫子曰:‘君子学道则爱人,小人学道则易使也。’”子曰:“二三子!偃之言是也。前言戏之耳。”

公山弗扰以费叛,召,子欲往。子路不说,曰:“末之也,已,何必公山氏之之也?”子曰:“夫召我者,而岂徒哉?如有用我者,吾其为东周乎?”

子张问仁于孔子。孔子曰:“能行五者于天下为仁矣。”“请问之。”曰:“恭,宽,信,敏,惠。恭则不侮,宽则得众,信则人任焉,敏则有功,惠则足以使人。”

佛肸召,子欲往。子路曰:“昔者由也闻诸夫子曰:‘亲于其身为不善者,君子不入也。’佛肸以中牟畔,子之往也,如之何?”子曰:“然,有是言也。不曰坚乎,磨而不磷;不曰白乎,涅而不缁。吾岂匏瓜也哉?焉能系而不食?”

子曰:“由也!女闻六言六蔽矣乎?”对曰:“未也。”“居!吾语女。好仁不好学,其蔽也愚;好知不好学,其蔽也荡;好信不好学,其蔽也贼;好直不好学,其蔽也绞;好勇不好学,其蔽也乱;好刚不好学,其蔽也狂。”

子曰:“小子何莫学夫诗?诗,可以兴,可以观,可以群,可以怨。迩之事父,远之事君;多识于鸟兽草木之名。”

子谓伯鱼曰:“女为《周南》、《召南》矣乎?人而不为《周南》、《召南》,其犹正墙面而立也与?”

子曰:“礼云礼云,玉帛云乎哉?乐云乐云,钟鼓云乎哉?”

子曰:“色厉而内荏,譬诸小人,其犹穿窬之盗也与?”

子曰:“乡愿,德之贼也。”

子曰:“道听而途说,德之弃也。”

子曰:“鄙夫可与事君也与哉?其未得之也,患得之。既得之,患失之。苟患失之,无所不至矣。”

子曰:“古者民有三疾,今也或是之亡也。古之狂也肆,今之狂也荡;古之矜也廉,今之矜也忿戾;古之愚也直,今之愚也诈而已矣。”

子曰:“巧言令色,鲜矣仁。”

子曰:“恶紫之夺朱也,恶郑声之乱雅乐也,恶利口之覆邦家者。”

子曰:“予欲无言。”子贡曰:“子如不言,则小子何述焉?”子曰:“天何言哉,四时行焉,百物生焉。天何言哉。”

孺悲欲见孔子,孔子辞以疾。将命者出户,取瑟而歌,使之闻之。

宰我问:“三年之丧,期已久矣。君子三年不为礼,礼必坏;三年不为乐,乐必崩。旧谷既没,新谷既升,钻燧改火,期可已矣。”子曰:“食夫稻,衣夫锦,于女安乎?”曰:“安。”“女安则为之。夫君子之居丧,食旨不甘,闻乐不乐,居处不安,故不为也。今女安,则为之。”宰我出。子曰:“予之不仁也。子生三年,然后免于父母之怀。夫三年之丧,天下之通丧也。予也有三年之爱于其父母乎?”

子曰:“饱食终日,无所用心,难矣哉!不有博弈者乎?为之,犹贤乎已。”

子路曰:“君子尚勇乎?”子曰:“君子义以为上,君子有勇而无义为乱,小人有勇而无义为盗。”

子贡曰:“君子亦有恶乎?”子曰:“有恶:恶称人之恶者,恶居下流而讪上者,恶勇而无礼者,恶果敢而窒者。”曰:“赐也亦有恶乎?”“恶徼以为知者,恶不孙以为勇者,恶讦以为直者。”

子曰:“唯君子与小人为难养也,近之则不孙,远之则怨。”

子曰:“年四十而见恶焉,其终也已。”


\chapter{微子第十八}\label{ch18}
%《论语·微子》



微子去之,箕子为之奴,比干谏而死。孔子曰:“殷有三仁焉。”

柳下惠为士师,三黜。人曰:“子未可以去乎?”曰:“直道而事人,焉往而不三黜?枉道而事人,何必去父母之邦?”

齐景公待孔子曰:“若季氏,则吾不能;以季孟之间待之。”曰:“吾老矣,不能用也。”孔子行。

齐人归女乐,季桓子受之,三日不朝,孔子行。

楚狂接舆歌而过孔子曰:“凤兮凤兮!何德之衰?往者不可谏,来者犹可追。已而,已而!今之从政者殆而!”孔子下,欲与之言。趋而辟之,不得与之言。

长沮、桀溺耦而耕,孔子过之,使子路问津焉。长沮曰:“夫执舆者为谁?”子路曰:“为孔氏。”曰:“是鲁孔丘与?”曰:“是也。”曰:“是知津矣。”问于桀溺。桀溺曰:“子为谁?”曰:“为仲由。”曰:“是鲁孔丘之徒与?”对曰:“然。”曰:“涛涛者天下皆是也,而谁以易之?且而与其从辟人之士也,岂若从辟世之士哉?”櫌而不辍。子路行以告。夫子怃然曰:“鸟兽不可与同群,吾非斯人之徒与而谁与?天下有道,丘不与易也。”

子路从而后,遇丈人,以杖荷蓧。子路问曰:“子见夫子乎?”丈人曰:“四体不勤,五谷不分。孰为夫子?”植其杖而芸。子路拱而立。止子路宿,杀鸡为黍而食之,见其二子焉。明日,子路行以告。子曰:“隐者也。”使子路反见之。至,则行矣。子路曰:“不仕无义。长幼之节,不可废也;君臣之义,如之何其废之?欲洁其身,而乱大伦。君子之仕也,行其义也。道之不行,已知之矣。”

逸民:伯夷、叔齐、虞仲、夷逸、朱张、柳下惠、少连。子曰:“不降其志,不辱其身,伯夷、叔齐与!”谓“柳下惠、少连,降志辱身矣,言中伦,行中虑,其斯而已矣。”谓“虞仲、夷逸,隐居放言,身中清,废中权。我则异于是,无可无不可。”

大师挚适齐,亚饭干适楚,三饭缭适蔡,四饭缺适秦,鼓方叔入于河,播鼗武入于汉,少师阳、击磬襄入于海。

周公谓鲁公曰:“君子不施其亲,不使大臣怨乎不以。故旧无大故,则不弃也。无求备于一人!”

周有八士:伯达、伯适、仲突、仲忽、叔夜、叔夏、季随、季騧。


\chapter{子张第十九}\label{ch19}
%《论语·子张》



子张曰:“士见危致命,见得思义,祭思敬,丧思哀,其可已矣。”

子张曰:“执德不弘,信道不笃,焉能为有?焉能为亡?”

子夏之门人问交于子张。子张曰:“子夏云何?”对曰:“子夏曰:‘可者与之,其不可者拒之。’”子张曰:“异乎吾所闻:君子尊贤而容众,嘉善而矜不能。我之大贤与,于人何所不容?我之不贤与,人将拒我,如之何其拒人也?”

子夏曰:“虽小道,必有可观者焉;致远恐泥,是以君子不为也。”

子夏曰:“日知其所亡,月无忘其所能,可谓好学也已矣。”

子夏曰:“博学而笃志,切问而近思,仁在其中矣。”

子夏曰:“百工居肆以成其事,君子学以致其道。”

子夏曰:“小人之过也必文。”

子夏曰:“君子有三变:望之俨然,即之也温,听其言也厉。”

子夏曰:“君子信而后劳其民;未信,则以为厉己也。信而后谏;未信,则以为谤己也。”

子夏曰:“大德不逾闲,小德出入可也。”

子游曰:“子夏之门人小子,当洒扫应对进退,则可矣,抑末也。本之则无,如之何?”子夏闻之,曰:“噫!言游过矣!君子之道,孰先传焉?孰后倦焉?譬诸草木,区以别矣。君子之道,焉可诬也?有始有卒者,其惟圣人乎!”

子夏曰:“仕而优则学,学而优则仕。”

子游曰:“丧至乎哀而止。”

子游曰:“吾友张也为难能也,然而未仁。”

曾子曰:“堂堂乎张也,难与并为仁矣。”

曾子曰:“吾闻诸夫子:人未有自致者也,必也亲丧乎!”

曾子曰:“吾闻诸夫子:孟庄子之孝也,其他可能也;其不改父之臣与父之政,是难能也。”

孟氏使阳肤为士师,问于曾子。曾子曰:“上失其道,民散久矣。如得其情,则哀矜而勿喜。”

子贡曰:“纣之不善,不如是之甚也。是以君子恶居下流,天下之恶皆归焉。”

子贡曰:“君子之过也,如日月之食焉:过也,人皆见之;更也,人皆仰之。”

卫公孙朝问于子贡曰:“仲尼焉学?”子贡曰:“文武之道,未坠于地,在人。贤者识其大者,不贤者识其小者。莫不有文武之道焉。夫子焉不学?而亦何常师之有?”

叔孙武叔语大夫于朝曰:“子贡贤于仲尼。”子服景伯以告子贡。子贡曰:“譬之宫墙,赐之墙也及肩,窥见室家之好。夫子之墙数仞,不得其门而入,不见宗庙之美,百官之富。得其门者或寡矣。夫子之云,不亦宜乎!”

叔孙武叔毁仲尼。子贡曰:“无以为也!仲尼不可毁也。他人之贤者,丘陵也,犹可逾也;仲尼,日月也,无得而逾焉。人虽欲自绝,其何伤于日月乎?多见其不知量也。”

陈子禽谓子贡曰:“子为恭也,仲尼岂贤于子乎?”子贡曰:“君子一言以为知,一言以为不知,言不可不慎也。夫子之不可及也,犹天之不可阶而升也。夫子之得邦家者,所谓立之斯立,道之斯行,绥之斯来,动之斯和。其生也荣,其死也哀,如之何其可及也?”


\chapter{尧曰第二十}\label{ch20}
%《论语·尧曰》



尧曰:“咨!尔舜!天之历数在尔躬,允执厥中。四海困穷,天禄永终。”
舜亦以命禹。曰:“予小子履敢用玄牡,敢昭告于皇皇后帝:有罪不敢赦。帝臣不蔽,简在帝心。朕躬有罪,无以万方;万方有罪,罪在朕躬。”

周有大赉,善人是富。“虽有周亲,不如仁人。百姓有过,在予一人。”

谨权量,审法度,修废官,四方之政行焉。兴灭国,继绝世,举逸民,天下之民归心焉。

所重:民、食、丧、祭。

子张问于孔子曰:“何如斯可以从政矣?”子曰:“尊五美,屏四恶,斯可以从政矣。”子张曰:“何谓五美?”子曰:“君子惠而不费,劳而不怨,欲而不贪,泰而不骄,威而不猛。”子张曰:“何谓惠而不费?”子曰:“因民之所利而利之,斯不亦惠而不费乎?择可劳而劳之,又谁怨?欲仁而得仁,又焉贪?君子无众寡,无小大,无敢慢,斯不亦泰而不骄乎?君子正其衣冠,尊其瞻视,俨然人望而威之,斯不亦威而不猛乎?”子张曰:“何谓四恶?”子曰:“不教而杀谓之虐;不戒视成谓之暴;慢令致期谓之贼;犹之与人也,出纳之吝谓之有司。”

孔子曰:“不知命,无以为君子也;不知礼,无以立也;不知言,无以知人也。”

\end{document}